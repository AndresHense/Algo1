
\documentclass[a4paper]{article}
\usepackage{a4wide}
%\usepackage{amsmath, amscd}
\usepackage[spanish,activeacute]{babel}
\usepackage{enumerate}
\usepackage{xspace}
\usepackage{longtable}
\usepackage{graphics}
\usepackage{listings}


\usepackage{calc}
\usepackage{lmodern}
\usepackage{amssymb}
\usepackage{amsmath}
\usepackage{mathdots}
\usepackage{mathtools}
\usepackage{multicol}
\usepackage{enumitem}
\usepackage{tasks}


\input{../../macros/Algo1Macros}


\title{Resuelto Entregable Algoritmos y Estructuras de Datos I}
\author{Andres M. Hense,Victoria Espil}
\date{} % Activate to display a given date or no date (if empty),
% otherwise the current date is printed 

\begin{document}
%\maketitle


%\section*{Práctica 1 --- Lógica}

\practica{2}{Ejercicios Entregables Resueltos}

\begin{center}
\textbf{Integrantes:} Andrés M. Hense, Victoria Espil
\end{center}

\paragraph{Ejercicio 4.j} Escriba los siguientes predicados auxiliares sobre secuencias de enteros, aclarando los tipos de los parámetros que recibe:

\begin{itemize}
	\item $sinRepetidos,$ que determina si la secuencia no tiene repetidos.
\end{itemize}

\subsection*{Respuesta}

\begin{itemize}
	\item pred sinRepetidos$(S: seq\langle \mathbb{Z} \rangle)\{$\\
\hspace*{6mm}$(\forall i,j:\mathbb{Z})(0\leq i\wedge i<\vert S\vert \wedge 0\leq j\wedge j<\vert S\vert)\rightarrow_L (S[j] \neq S[i])$\\
$\}$
\end{itemize}

\paragraph{Ejercicio 8.c} Sean $P(x:\mathbb{Z})$ y $Q(x:\mathbb{Z})$ dos predicados cualesquiera que nunca se indefinen, sea $s$ una secuencia de enteros y sean $a,b$ y $k$ enteros. Decidir en cada caso la relación de fuerza entre las dos fórmulas.

\begin{itemize}
	\item $(\forall n:\mathbb{Z})((n \in s \wedge P(n))\rightarrow Q(n))$ y $(\forall n: \mathbb{Z})((n \in s)\rightarrow Q(n))$
\end{itemize}

\subsection*{Respuesta}

\begin{itemize}
\item $(\forall n: \mathbb{Z})((n \in s)\rightarrow Q(n))$ es mas fuerte.\\
\end{itemize}
\paragraph*{Explicacion:}
\begin{flushleft}
$A= (\forall n:\mathbb{Z})((n \in s \wedge P(n))\rightarrow Q(n))$\\
$B=(\forall n: \mathbb{Z})((n \in s)\rightarrow Q(n))$\\
\end{flushleft} 

Puede pasar que $A$ sea $True$ y $B$ sea $False$? (es decir que $A \rightarrow B$ sea $False$).\smallskip \\ 
Si, por ejemplo si $n \in S$ pero $\neg P(n)$ y $\neg Q(n)$. (El antecedente de $A$ sería $False$, y por lo tanto la implicación de $A$ sería $True$, pero en el $B$ el antecedente sería $True$ y el consecuente $False$).\par \medskip
Puede pasar que $B$ sea $True$ y $A$ sea $False$?.\smallskip \\
NO, Las únicas formas en las que $B$ fuera $True$ serian:\medskip \\
\textbf{caso 1:} $n\not\in s$, pero en ese caso $A$ también sería $True$.\smallskip \\
\textbf{caso 2:} $n\in s$ y $Q(n)$, pero en ese caso el consecuente de $A$ sería siempre verdadero, entonces $A$ no podria ser $False$.




\paragraph{Ejercicio 14.e} Sea m una secuencia de secuencias de tipo $\mathbb{Z}$, escribir una expresión tal que:

\begin{itemize}
	\item Retorne la suma de todas las posiciones impares de cada secuencia.
\end{itemize}

\subsection*{Respuesta}

\begin{itemize}
\item \textit{aux sumaPosicionesImpares}$(m: seq\langle seq\langle \mathbb{Z}\rangle \rangle):\mathbb{Z}
	=\sum_{i=0}^{|m|-1}\sum_{j=0}^{|m[i]|-1} IfThenElseFi(j\textrm{ mod }2=1,m[i][j],0)$
\end{itemize}

\end{document}