
\documentclass{article}

\usepackage{a4wide}
\usepackage[utf8]{inputenc}
\usepackage{enumerate}
\usepackage{xspace}
\usepackage{longtable}
\usepackage{graphics}
\usepackage{listings}
\usepackage{xcolor}
\usepackage{calc}
\usepackage{lmodern}
\usepackage{amssymb}
\usepackage{amsmath}
\usepackage{mathdots}
\usepackage{mathtools}
\usepackage{multicol}
\usepackage{enumitem}
\usepackage{tasks}
\usepackage{pdfpages}

\input{../macros/Algo1Macros}

\begin{document}

%\practica{4}{Precondición más débil en SmallLang}
\includepdf[pages=-]{practica5.pdf}

\begin{center}
\section*{RESOLUCIONES.}
\end{center}

\paragraph{Ejercicio 1}
\begin{enumerate}[label=\alph*)]
\item
	\begin{align*}
		P_c&: \{result=0\wedge i=0\}\\
		Q_c&: \{result=\sum_{j=0}^{|s|-1}s[j]\}
	\end{align*}
\item Si le sacas el igual a la segunda desigualdad no te agarra el ultimo elemento la 
	sumatoria.
\item Si le sacas el -1 al supraindice de la sumatoria entonces va a tratar de sumar la posicion del tamaño
	de la secuencia, como la secuencia no posee esta posicion se va a indefinir el resultado.
\item Si invertis de lugar las instrucciones del cuerpo del ciclo entonces no te agarra el primer elemento de la 
secuencia.
\item 
	Vamos  a demostrar los primeros 3 puntos del teorema del invariante. Esto nos permite afirmar que el ciclo, si termina entonces
	es correcto respecto a su especificacón. Especificacion del ciclo:
	\begin{itemize}
		\item $P_c: result=0 \wedge i=0$
		\item $Q_c:r esult=\sum_{j=0}^{|s|-1}s[j] $
		\item $I\equiv 0\leq i\leq |s| \wedge_L result=\sum_{j=0}^
			{i-1}s[j]$
	\end{itemize}
	\begin{flushleft}$P_c\Rightarrow I$\end{flushleft}
	Tenemos que vale $P_c$ como hipótesis. Queremos probar que vale I. Vamos a probarlo por partes:
	\begin{itemize}
		\item Queremos ver que vale $ 0\leq i\leq |s|$\smallskip \\
		Sabemos que i=0 (es información que nos da $P_c$). Entonces reemplazando, lo que queremos ver es que 
		$0\leq 0\leq |s|$. La primera parte, $0\leq 0$, es una tautología. Nos resta ver si $0\leq |s|$. Pero si la lista es
		vacia entonces $0\leq 0$, caso contrario si la lista fuera no vacia entonces $|s|>0$ y la desigualdad $0\leq |s|$ 
		se cumple.
		\item Queremos ver que $result=\sum_{j=0}^{i-1}s[j]$\\
		Como $i=0$, entonces $\sum_{j=0}^{i-1}s[j]=0=result$.
	\end{itemize}	 
	$(I\wedge \neg B)\Rightarrow Q_c$\smallskip \\
	Queremos demostrar que vale $Q_c$, asumiendo que valen tanto $I$ como $\neg B$.\\ Es decir, queremos probar que 
	$result =\sum_{j=0}^{|s|-1}s[j]$\\
	Como sabemos que vale $I$, podemos afirmar que $0\leq i\leq |s|$.\\ Además sabemos que vale $\neg B\equiv i\geq |s|$.
	\\ Luego $|s|\leq i\leq |s|$, entonces el unico valor que cumple esta condición es $i=|s|$. Analicemoslo.\\
	\begin{itemize}
	\item $i=|s|$
	\end{itemize}
	En este caso, podemos reemplazar $i$ por $|s|$ en la sumatoria del invariante y llegamos a $result =\sum_{j=0}^{|s|-1}s[j]$, exactamante lo que queríamos probar.
	$\{I\wedge B\} \textbf{ ciclo } \{I\}$\smallskip \\
	Queremos ver que vale la siguiente tripla de Hoare $\{I\wedge B\} \textbf{ ciclo } \{I\}$.\\
	Llamemos S1 a la primer instrucción del cuerpo del ciclo, S2 a la segunda:\\
	S1$: result:=result+s[i];$\\
	S2$: i:=i+1$\\
	Lo primero que haremos es calcular $wp(ciclo,I)$.
	\begin{equation}wp(S1;S2,I)\stackrel{Ax3}{\equiv}wp(S1,wp(S2,I))\end{equation}
	Antes de seguir, debemos calcular $wp(S2,I)$. Para eso usaremos el axioma 1:
	\begin{align*}
	wp(S2,I)&\stackrel{Ax1}{\equiv}def(i+1)\wedge_L I_{i+1}^{i}\\
		&true\wedge_L (0\leq i+1\leq |s| \wedge_L result=\sum_{j=0}^{i+1-1}s[j])\\
		& (0\leq i+1\leq |s| \wedge_L result=\sum_{j=0}^{i+1-1}s[j])
	\end{align*}
	Volviendo a (1), reemplazamos $wp(S2,I)$ y nos queda:
	\begin{align*}
	wp(S1;S2,I)&\stackrel{Ax3}{\equiv}wp(S1,wp(S2,I))\equiv wp(S1,0\leq i+1\leq |s| \wedge_L result=\sum_{j=0}^{i+1-1}s[j])\\
		&\stackrel{Ax1}{\equiv}def(result+s[i])\wedge_L 0\leq i+1\leq |s| \wedge_L result+s[i]=\sum_{j=0}^{i}s[j]\\
		&\equiv 0\leq i<|s|\wedge_L 0\leq i+1\leq |s| \wedge_L result+s[i]=\sum_{j=0}^{i}s[j]\\
		&\equiv 0\leq i \wedge i+1\leq |s| \wedge_L result=\sum_{j=0}^{i}s[j]-s[i]\\
		&\equiv 0\leq i \wedge i+1\leq |s| \wedge_L result=\sum_{j=0}^{i-1}s[j]
	\end{align*}
	Una vez calculada la precondición más débil, debemos ver si $(I\wedge B)$ implican dicha precondición. Probamos cada
	parte por separado:
	\begin{itemize}
		\item $ 0\leq i \wedge i+1\leq |s|$\smallskip \\
		Sabemos por $I$ que $i\leq 0$, entonces la primera parte ya esta demostrada, ahora AAAAAAAHHHHHH
		\item $result=\sum_{j=0}^{i-1}s[j]$\smallskip \\
		Este $result$ es igual al del $I$.
	\end{itemize}
\end{enumerate}

%\includegraphics[width=\linewidth]{pe.png}


\end{document}