
\documentclass[a4paper]{article}
\usepackage{a4wide}
\usepackage[spanish,activeacute]{babel}
\usepackage{enumerate}
\usepackage{xspace}
\usepackage{longtable}
\usepackage{graphics}
\usepackage{listings}


\usepackage{calc}
\usepackage{lmodern}
\usepackage{amssymb}
\usepackage{amsmath}
\usepackage{mathdots}
\usepackage{mathtools}
\usepackage{multicol}
\usepackage{enumitem}
\usepackage{tasks}


\input{../../macros/Algo1Macros}


\title{Resuelto Entregable Algoritmos y Estructuras de Datos I}
\author{Andres M. Hense,Victoria Espil}
\date{}
\begin{document}
%\maketitle


%\section*{Práctica 1 --- Lógica}

\practica{4}{Resolución de los Ejercicios Entregables}

\begin{center}
\textbf{Integrantes:} Andrés M. Hense, Victoria Espil
\end{center}

\paragraph{Ejercicio 12}
Para probar que el programa es correcto respecto a la especificacón, vamos a probar estas implicaciones
 por separado, y por monotonia llegaremos a que el programa es correcto.\\
\begin{itemize}
\item $Pre \rightarrow wp(\textbf{codigo previo al ciclo},P_c)$
\item $P_c \rightarrow wp(\textbf{ciclo},Q_c)$
\item $Q_c \rightarrow wp(\textbf{codigo posterior al ciclo},Post)$
\end{itemize}
 Especificacion del ciclo:
	\begin{itemize}
		\item $Pre: True$
		\item $Post: r=True \leftrightarrow ((\exists k:\mathbb{Z})(0\leq k<|s|)\wedge_L s[k=e]) $
		\item $P_c: i=0\wedge j=-1$
		\item $Q_c: i=|s|\wedge (j!=-1)\leftrightarrow ((\exists k:\mathbb{Z})(0\leq k<|s|)\wedge_L s[k=e])$
		\item $B: i<|s|$
		\item $I: 0\leq i<|s|\wedge (j!=-1)\leftrightarrow ((\exists k:\mathbb{Z})(0\leq k<i)\wedge_L s[k=e]) $
		\item $f_v:|s|-i$
	\end{itemize}
Empecemos probando la primer implicación\\
$\scalebox{1.3}{$Pre\rightarrow wp(\textbf{codigo previo al ciclo},P_c)$}$\medskip \\
\begin{align*}
wp(i:=0; j:=-1,P_c)&\equiv wp(i:=0,wp(j:=-1,P_c))\\
&\equiv wp(i:=0,(P_c)_{-1}^{j})\\
&\equiv (i=0\wedge -1=-1)_{0}^{i}\\
&\equiv 0=0\wedge True\\
&\equiv True
\end{align*}
Luego $True\rightarrow True$, es tautologia.\medskip\\
$\scalebox{1.3}{$P_c \rightarrow wp(\textbf{ciclo},Q_c)$}$\medskip \\

Demostraremos que el ciclo es correcto respecto a su especificación y ademas termina.
	\begin{flushleft}$\scalebox{1.3}{$P_c\Rightarrow I$}$\end{flushleft}
	Debemos demostrar que vale $I$ sabiendo que vale $P_c$ 
	\begin{itemize}
		\item $ 0\leq i< |s|$\smallskip \\
		Por $P_c$ sabemos que $i=0$, entonces $0\leq i $ vale. Además, $|s|\geq 0$ (porque las listas no pueden tener una cantidad 
		negativa de elementos), luegos $|s|\geq i$
		\item $ (j!=-1)\leftrightarrow ((\exists k:\mathbb{Z})(0\leq k<i)\wedge_L s[k=e])$\smallskip \\
		$P_c$ indica que $i=0$ y $j=-1$. Luego como no existe un $k$ entre $0\leq k <0$, la doble implicacion no se cumple, por lo
		 que tiene que pasar que $j=-1$.
	\end{itemize}	 
	$\scalebox{1.3}{$(I\wedge \neg B)\Rightarrow Q_c$}$\smallskip \\
	Queremos demostrar que vale $Q_c$, asumiendo que vale  $I\wedge \neg B$.
	\begin{itemize}
	\item Por $I$ sabemos que $i< |s|$, y por $\neg B$ sabemos que $i\geq |s|$. Entonces $i$ debe ser igual a $|s|$
	\item Es trivial ver que vale $ (j!=-1)\leftrightarrow ((\exists k:\mathbb{Z})(0\leq k<|s|)\wedge_L s[k=e])$ ya que, al reemplazar
	 $i=|s|$ en el invariante llego a eso.
	\end{itemize}
	$\scalebox{1.3}{$\{I\wedge B\} \textbf{ ciclo } \{I\}$}$\medskip \\
	Queremos ver que vale la siguiente tripla de Hoare $\{I\wedge B\} \textbf{ ciclo } \{I\}$.\\
	Llamemos S1 a la primer instrucción del cuerpo del ciclo, S2 a la segunda:\\
	S1$: r[i]:=s[i];$\\
	S2$: i:=i+1$\\
	Lo primero que haremos es calcular $wp(ciclo,I)$.
	\begin{equation}wp(S1;S2,I)\stackrel{Ax3}{\equiv}wp(S1,wp(S2,I))\end{equation}
	
	$\scalebox{1.3}{$\{(I\wedge B\wedge v_0=f_v)\}\textbf{ ciclo }\{(f_v<v_0)\}$}$\medskip \\
	Dado que queremos demostrar que vale una tripla de Hoare, comenzaremos calculando la\\ precondición más débil $wp(ciclo,f_v<v_0)$.
	\begin{align*}
		wp(S1;S2,f_v<v_0)&\stackrel{Ax3}{\equiv}wp(S1,wp(S2,|s|-i<v_0))\\
			&\stackrel{Ax1}{\equiv}wp(S1,true\wedge_L |s|-(i+1)<v_0)\\
			&\stackrel{Ax3}{\equiv}true\wedge_L(true\wedge_L  |s|-(i+1)<v_0)\\
			&\equiv |s|-i-1<v_0
	\end{align*}
	Es decir, $wp(S1;S2,f_v<v_0)=|s|-i-1<v_0$. Ahora debemos ver que $(I\wedge B\wedge v_0=f_v)$ implican dicha WP. Parte de la 
	hipótesis es que $v_0=f_v$, es decir $v_0=|s|-i$. Restando 1 a ambos lados se prueba, $|s|-i-1=v_0-1<v_0$.\medskip \\
	$\scalebox{1.3}{$(I\wedge f_v\leq 0)\Rightarrow \neg B$}$\medskip\\
	Debemos mostrar que vale $\neg B$, es decir $i\geq |s|$.\\
	Sabemos que $f_v\leq 0$, es decir $|s|-i\leq 0$, luego $|s|\leq i$, como queriamos demostrar.\medskip\\
	\noindent
$\scalebox{1.3}{$Q_c \rightarrow wp(\textbf{codigo posterior al ciclo},Post)$}$\medskip \\
\textbf{S: if $(j != -1)$ then $r=True$ else $r=False$ endif}\\
\begin{align*}
   			wp(\textbf{S},Post)&\equiv \textrm{def}(j != -1)\wedge_L 
   				\Bigg(\Big((j!=-1)\wedge wp(r=True,Post))\Big) \vee\Big(\neg (j!=-1)\wedge wp(j=False,Post)\Big)\Bigg)\\
   					&\equiv True \wedge \Bigg(\Big((j!=-1)\wedge Post_{True}^{r})\Big) 
   						\vee\Big(j=-1\wedge Post_{False}^{r})\Big)\Bigg)\\ 
   				&\equiv \Big(j!=-1\wedge True=True \leftrightarrow ((\exists k:\mathbb{Z})(0\leq k<|s|)\wedge_L s[k=e])\Big) \vee\\
   						&\ \ \ \ \Big(j=-1\wedge False=True \leftrightarrow ((\exists k:\mathbb{Z})(0\leq k<|s|)\wedge_L s[k=e]))\Big)\\ 
   				&\equiv \Big(j!=-1\wedge True \leftrightarrow ((\exists k:\mathbb{Z})(0\leq k<|s|)\wedge_L s[k=e])\Big) \vee\\
   						&\ \ \ \ \Big(j=-1\wedge False \leftrightarrow ((\exists k:\mathbb{Z})(0\leq k<|s|)\wedge_L s[k=e]))\Big)\\   
   				&\equiv \Big(j!=-1\wedge ((\exists k:\mathbb{Z})(0\leq k<|s|)\wedge_L s[k=e])\Big) \vee\\
   						&\ \ \ \ \Big(j=-1\wedge \neg((\exists k:\mathbb{Z})(0\leq k<|s|)\wedge_L s[k=e]))\Big)\\    	
   				&\equiv (j!=-1)\leftrightarrow ((\exists k:\mathbb{Z})(0\leq k<|s|)\wedge_L s[k=e])  						   				  		
   		\end{align*}
Como llegue  a lo mismo que $Q_c$, entonces vale.

\end{document}