
\documentclass[a4paper]{article}
\usepackage{a4wide}
\usepackage[spanish,activeacute]{babel}
\usepackage{enumerate}
\usepackage{xspace}
\usepackage{longtable}
\usepackage{graphics}
\usepackage{listings}


\usepackage{calc}
\usepackage{lmodern}
\usepackage{amssymb}
\usepackage{amsmath}
\usepackage{mathdots}
\usepackage{mathtools}
\usepackage{multicol}
\usepackage{enumitem}
\usepackage{tasks}


\input{../../macros/Algo1Macros}


\title{Resuelto Entregable Algoritmos y Estructuras de Datos I}
\author{Andres M. Hense,Victoria Espil}
\date{}
\begin{document}
%\maketitle


%\section*{Práctica 1 --- Lógica}

\practica{4}{Resolución de los Ejercicios Entregables}

\begin{center}
\textbf{Integrantes:} Andrés M. Hense, Victoria Espil
\end{center}

\paragraph{Ejercicio 12}
Para probar que el programa es correcto respecto a la especificacón, vamos a probar estas implicaciones
 por separado, y por monotonia llegaremos a que el programa es correcto.\\
\begin{itemize}
\item $Pre \rightarrow wp(\textbf{codigo previo al ciclo},P_c)$
\item $P_c \rightarrow wp(\textbf{ciclo},Q_c)$
\item $Q_c \rightarrow wp(\textbf{codigo posterior al ciclo},Post)$
\end{itemize}
 Especificacion del ciclo:
	\begin{itemize}
		\item $P_c: i=0\wedge j=-1$
		\item $Q_c: r=True \leftrightarrow ((\exists k:\mathbb{Z})(0\leq k<|s|)\wedge_L s[k=e])$
		\item $B: i<|s|$
		\item $I: 0\leq i<|s|\wedge\ if\ (\exists k:\mathbb{Z})(0\leq k<i)\wedge_L s[k=e])\ then\ j=k\ else\ j=-1\ fi$
		\item $f_v:|s|-i$
	\end{itemize}
Empecemos probando la primer implicación\\
$\scalebox{1.3}{$Pre\rightarrow wp(\textbf{codigo previo al ciclo},P_c)$}$\medskip \\
\begin{align*}
wp(i:=0; j:=-1,P_c)&\equiv wp(i:=0,wp(j:=-1,P_c))\\
&\equiv wp(i:=0,(P_c)_{-1}^{j})\\
&\equiv (i=0\wedge -1=-1)_{0}^{i}\\
&\equiv 0=0\wedge True\\
&\equiv True
\end{align*}
Luego $True\rightarrow True$, es tautologia.\medskip\\
$\scalebox{1.3}{$P_c \rightarrow wp(\textbf{ciclo},Q_c)$}$\medskip \\

blablalbabla\medskip\\
$\scalebox{1.3}{$Q_c \rightarrow wp(\textbf{codigo posterior al ciclo},Post)$}$\medskip \\

\end{document}