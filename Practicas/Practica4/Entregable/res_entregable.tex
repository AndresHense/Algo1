
\documentclass[a4paper]{article}
\usepackage{a4wide}
\usepackage[spanish,activeacute]{babel}
\usepackage{enumerate}
\usepackage{xspace}
\usepackage{longtable}
\usepackage{graphics}
\usepackage{listings}


\usepackage{calc}
\usepackage{lmodern}
\usepackage{amssymb}
\usepackage{amsmath}
\usepackage{mathdots}
\usepackage{mathtools}
\usepackage{multicol}
\usepackage{enumitem}
\usepackage{tasks}


\input{../../macros/Algo1Macros}


\title{Resuelto Entregable Algoritmos y Estructuras de Datos I}
\author{Andres M. Hense,Victoria Espil}
\date{}
\begin{document}
%\maketitle


%\section*{Práctica 1 --- Lógica}

\practica{4}{Resolución de los Ejercicios Entregables}

\begin{center}
\textbf{Integrantes:} Andrés M. Hense, Victoria Espil
\end{center}

\paragraph{Ejercicio 1.} Calcular las siguientes expresiones, donde $a, b$ son variables 
reales, $i$ una variable entera y $A$ es una secuencia
de reales.
\begin{itemize}
\item def$( \sqrt{a/b}).$
\item def$(A[i + 2]).$
\end{itemize}

\subsection*{Respuesta:}
Supongo que def$(x)\equiv True$, para todas las variables por lo expuesto en la teorica;
ya que de este\\ \hspace*{4mm} modo se simplifica la notación.

\begin{itemize}
  \item def$( \sqrt{a/b})\equiv b\neq 0\wedge (a/b)\geq 0$.
  \item def$(A[i + 2])\equiv 0\leq i+2 < |A|$
\end{itemize}

\paragraph{Ejercicio 6.e}Escribir programas para los siguientes problemas y demostrar 
formalmente su corrección usando la precondición
más débil.

\begin{itemize}
\item \textbf{proc problema5 }(in a$: seq\langle \mathbb{Z}\rangle$, in i$:\mathbb{Z}$
	, out result$: \mathbb{Z}$) {\\
   \hspace*{6mm} \textbf{Pre }$\{0 \leq i \wedge i + 1 < |a|\}$\\
   \hspace*{6mm} \textbf{Post }$\{result = a[i] + a[i + 1]\}$\\
   }
\end{itemize}

\subsection*{Respuesta:}

\begin{enumerate}

\item Calculamos $\color{blue}\{wp(S,Post)\}$
		\begin{align*}
		\color{blue}\{wp(S,Post)\}
			&\color{blue} \equiv \{\textrm{def}(a[i] + a[i + 1])
				\wedge_L a[i] + a[i + 1]=a[i] + a[i + 1] \}\\
			&\color{blue} \equiv \textrm{def}(a[i]) \wedge_L
				 \textrm{def}(a[i + 1]) \wedge_L 0\leq i \wedge i+1<|a|
				  \wedge a[i] + a[i + 1]=a[i] + a[i + 1]\\
			&\color{blue} \equiv True \wedge_L
				 True \wedge_L True \wedge_L 0\leq i \wedge i+1<|a|
				  \wedge True\\
			&\color{blue} \equiv 0\leq i \wedge i+1<|a| \\	  
		\textbf{S:} &\  result := a[i] + a[i + 1]\\
		\color{blue}\{\textbf{Post:} &\color{blue}\  result = a[i] + a[i + 1]\}\\
		\end{align*}
\item Chequeamos $Pre \rightarrow \color{blue}\{wp(S,Post)\}$		
		\begin{align*}
		Pre &\rightarrow \color{blue}\{wp(S,Post)\}\\
		\{0 \leq i \wedge i + 1 < |a|\}&\rightarrow \color{blue}\{0\leq i \wedge i+1<|a|\}\\
		& True 
		\end{align*}
\end{enumerate}
\paragraph{Ejercicio 8.} Escribir programas para los siguientes problemas y demostrar 
formalmente su corrección usando la precondición más débil

\begin{itemize}
\item \textbf{proc problema5 }(in s$: seq\langle \mathbb{Z}\rangle$, in i$:\mathbb{Z}$
	, inout a$: \mathbb{Z}$) \{\smallskip \\                            
    \hspace*{6mm} \textbf{Pre }$\{0 \leq i < |s| \wedge_L a=\sum_{j=0}^{i-1}($if $s[j] \neq 0$ 
    then 1 else 0 fi$)\}$\smallskip \\          
   \hspace*{6mm} \textbf{Post }$\{a=\sum_{j=0}^{i}($if $s[j] \neq 0$ 
    then 1 else 0 fi$)\}$\\
   \}     
\end{itemize}

\subsection*{Respuesta:}

\end{document}