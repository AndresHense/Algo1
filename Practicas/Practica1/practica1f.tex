% !TEX TS-program = pdflatex
% !TEX encoding = UTF-8 Unicode

% This is a simple template for a LaTeX document using the "article" class.
% See "book", "report", "letter" for other types of document.

% align equations to the left
% use larger type; default would be 10pt
%\documentclass[fleqn, 11pt]{article}
\documentclass[a4paper]{article}
\usepackage{a4wide}
%\usepackage{amsmath, amscd}
\usepackage[spanish,activeacute]{babel}
\usepackage{enumerate}
\usepackage{xspace}
\usepackage{longtable}
\usepackage{graphics}
\usepackage{listings}
%\usepackage[utf8]{inputenc} % set input encoding (not needed with XeLaTeX)

%%% Examples of Article customizations
% These packages are optional, depending whether you want the features they
% provide.
% See the LaTeX Companion or other references for full information.

%%% PAGE DIMENSIONS
%\usepackage{geometry} % to change the page dimensions
%\geometry{a4paper} % or letterpaper (US) or a5paper or....
% for example, change the margins to 2 inches all round
%\geometry{margin=12mm}
% set up the page for landscape
% \geometry{landscape}
%   read geometry.pdf for detailed page layout information

% support the \includegraphics command and options
%\usepackage{graphicx}

% Activate to begin paragraphs with an empty line rather than an indent
%\usepackage[parfill]{parskip}

%%% PACKAGES
% for much better looking tables
%\usepackage{booktabs}
% for better arrays (eg matrices) in maths
%\usepackage{array}
% very flexible & customisable lists (eg. enumerate/itemize, etc.)
%\usepackage{paralist}
% adds environment for commenting out blocks of text & for better verbatim
%\usepackage{verbatim}
% make it possible to include more than one captioned figure/table in a single
% float
%\usepackage{subfig}
% These packages are all incorporated in the memoir class to one degree or
% another...

%%% HEADERS & FOOTERS
%\usepackage{fancyhdr} % This should be set AFTER setting up the page geometry
%\pagestyle{fancy} % options: empty , plain , fancy
%\renewcommand{\headrulewidth}{0pt} % customise the layout...
%\lhead{}\chead{}\rhead{}
%\lfoot{}\cfoot{\thepage}\rfoot{}

%%% SECTION TITLE APPEARANCE
%\usepackage{sectsty}
% (See the fntguide.pdf for font help)
%\allsectionsfont{\sffamily\mdseries\upshape}
% (This matches ConTeXt defaults)

%%% ToC (table of contents) APPEARANCE
% Put the bibliography in the ToC
%\usepackage[nottoc, notlof, notlot]{tocbibind}
% Alter the style of the Table of Contents
%\usepackage[titles, subfigure]{tocloft}
%\renewcommand{\cftsecfont}{\rmfamily\mdseries\upshape}
%\renewcommand{\cftsecpagefont}{\rmfamily\mdseries\upshape} % No bold!

\usepackage{calc}
\usepackage{lmodern}
\usepackage{amssymb}
\usepackage{amsmath}
\usepackage{mathdots}
\usepackage{mathtools}
\usepackage{multicol}
\usepackage{enumitem}
\usepackage{tasks}

\input{../macros/Algo1Macros}

%\newcommand{\nat}{\mathbb{N}}
%\newcommand{\Ccur}{\mathcal{C}}
%\newcommand{\indef}{\uparrow}

%\overfullrule=2em

%%% END Article customizations

%%% The "real" document content comes below...

\title{Resueltos Algoritmos y Estructuras de Datos I}
\author{Andres M. Hense}
%\date{} % Activate to display a given date or no date (if empty),
% otherwise the current date is printed 

\begin{document}


\practica{1}{Lógica Resuelto}

\subsection*{Comentarios:}
Hola, este no es un resuelto oficial, tiene el logo del DC porque me parecio divertido copiar el formato de la guia, los ejercicios que dicen "Determinar blabla" deberian estar justificados.

\section{Lógica binaria (Verdadero o Falso)}

\paragraph{\textbf{Ejercicio 1.}} $\bigstar$ Sean p y q variables proposicionales. ¿Cuales de las siguientes expresiones son formulas bien formadas?

\begin{tasks}(3)
	\task $ (p \neg q) $
	\task[d)] $ \neg (p) $
	\task[g)] $ ( \neg p) $	
	\task $ p \bigvee q \bigwedge True $
	\task[e)] $ (p \bigvee \neg p \bigwedge q) $
	\task[h)] $ (p \bigvee False) $
	\task $ (p \rightarrow \neg q \rightarrow q) $		
	\task[f)] $ (True \bigwedge True \bigwedge True) $	
	\task[i)] $ (p=q) $ 
\end{tasks}

\subsection*{Respuesta}

\begin{tasks}(3)
	\task No está bien formada.
	\task[d)] No está bien formada.
	\task[g)] No está bien formada.
	\task No está bien formada.
	\task[e)] No está bien formada.
	\task[h)] Está bien formada.
	\task No está bien formada.
	\task[f)] Está bien formada.
	\task[i)] Está bien formada.
\end{tasks}

\paragraph{\textbf{Ejercicio 2.}} $\bigstar$  Sean $ x: \mathbb{Z} ,y: \mathbb{Z}$ y $ z: \textbf{Bool} $ tres variables. ¿Cuales de las siguientes expresiones pueden tiparse correctamente?


\begin{tasks}(2)
	\task $ (1=0) \vee (x=y) $
	\task[d)] $ z= true \leftrightarrow (y=x) $
	\task $ (x+10)=y $
	\task[e)] $ (z=0) \vee (z=1) $
	\task $ (x \vee y) $
	\task[f)] $ y + (y<0) $
\end{tasks}

\subsection*{Respuesta}

\begin{tasks}(2)
	\task Tipa.
	\task[d)] Tipa.
	\task Tipa.
	\task[e)] No tipa.
	\task No tipa.
	\task[f)] No tipa.
\end{tasks}

\paragraph{\textbf{Ejercicio 3.}} La fórmula $ (3+7= \pi -8) \wedge True $ es una fórmula bien formada ¿Por qué? Justifique informal, pero detalladamente, su respuesta.

\subsection*{Respuesta}

Lo que esta en el parentesis pregunta si esos dos números son iguales y devuelve un booleano al que se le aplica una AND con True, no tiene errores de tipo, por lo que esta bien formulado.

\paragraph{\textbf{Ejercicio 4.}} $\bigstar$ Determinar el valor de verdad de las siguientes proposiciones

\begin{multicols}{2}
\begin{enumerate}[label=\alph*)]
	\item $ ( \neg a \vee b) $
	\item $ (c \vee (y \wedge x) \vee b) $
	\item $ \neg (c \vee y) $
	\item $ (\neg(c \vee y) \leftrightarrow (\neg c \wedge \neg y)) $
	\item $ ((c \vee y) \wedge (x \vee b)) $
	\item $ (((c \vee y) \wedge (x \vee b)) \leftrightarrow (c \vee (y \wedge x)\vee b)) $
	\item $ (\neg c \wedge \neg y)$
\end{enumerate}
\end{multicols}

cuando el valor de verdad de a,b y c es verdadero, mientras que el de x e y es falso.

\subsection*{Respuesta}

\begin{multicols}{2}
\begin{enumerate}[label=\alph*)]
	\item $ True$
	\item $ True$
	\item $ False$
	\item $ True$
	\item $ True$
	\item $ True$
	\item $ False$
\end{enumerate}
\end{multicols}


\paragraph{\textbf{Ejercicio 5.}} Determinar, utilizando tablas de verdad, si las siguientes fórmulas son tautologias, contradicciones o contingencias.

\begin{multicols}{2}
\begin{enumerate}[label=\alph*)]
\item $ (p \vee \neg p)$
\item $ (p \wedge \neg p) $
\item $ ((\neg p \vee q) \leftrightarrow(p \rightarrow q))$
\item $ ((p \vee q) \rightarrow p)$
\item $ (\neg (p \wedge q) \leftrightarrow(\neg p \vee \neg q)) $ 
\item $ (p \rightarrow p) $
\item $ ((p \wedge q) \rightarrow p)$
\item $ ((p \wedge (q \vee r)) \leftrightarrow ((p \wedge q)\vee (p \wedge r)))$  
\item $ ((p \rightarrow (q \rightarrow r))\rightarrow ((p \rightarrow ) \rightarrow (p \rightarrow r))) $
              
\end{enumerate}
\end{multicols}

\subsection*{Respuesta}
Armar Tablas de la Verdad es un plomazo y no aporta en nada, chiau.

\begin{enumerate}[label=\alph*)]
\begin{multicols}{2}
\item 
\begin{tabular}{|c|c|c|}
\hline 
$p$ & $\neg p$ & $(p\vee \neg p)$ \\ 
\hline 
0 & 1 & 1 \\ 
\hline 
1 & 0 & 1 \\ 
\hline 
\end{tabular} 
Tautologia
\item
\begin{tabular}{|c|c|c|}
\hline 
$p$ & $\neg p$ & $(p\wedge \neg p)$ \\ 
\hline 
0 & 1 & 0 \\ 
\hline 
1 & 0 & 0 \\ 
\hline 
\end{tabular} 
Contradicción
\end{multicols}
\item
\begin{tabular}{|c|c|c|c|c|}
\hline 
p & q & $(\neg p \vee q)$ & $(p\rightarrow q)$ & $(\neg(p\wedge q)\leftrightarrow (\neg p \vee \neg q))$ \\ 
\hline 
0 & 0 & 1 & 1 & 1 \\ 
\hline 
0 & 1 & 1 & 1 & 1 \\ 
\hline 
1 & 0 & 0 & 0 & 1 \\ 
\hline 
1 & 1 & 1 & 1 & 1 \\ 
\hline 
\end{tabular} 
Tautologia
\item
\begin{tabular}{|c|c|c|c|}
\hline 
$p$ & $q$ & $(p\vee q)$ & $((p \vee q)\rightarrow) p$ \\ 
\hline 
0 & 0 & 0 & 1 \\ 
\hline 
0 & 1 & 1 & 0 \\ 
\hline 
1 & 0 & 1 & 1 \\ 
\hline 
1 & 1 & 1 & 1 \\ 
\hline 
\end{tabular} 
Contingencia
\item
\begin{tabular}{|c|c|c|c|c|}
\hline 
p & q & $\neg (p\wedge q)$ & $(\neg p \vee \neg q)$ & $(\neg (p\wedge q)\leftrightarrow (\neg p \vee \neg q))$ \\ 
\hline 
0 & 0 & 1 & 1 & 1 \\ 
\hline 
0 & 1 & 1 & 1 & 1 \\ 
\hline 
1 & 0 & 1 & 1 & 1 \\ 
\hline 
1 & 1 & 0 & 0 & 1 \\ 
\hline 
\end{tabular} 
Tautologia
\item
\begin{tabular}{|c|c|}
\hline 
$p$ & $(p\rightarrow p)$ \\ 
\hline 
0 & 1 \\ 
\hline 
1 & 1 \\ 
\hline 
\end{tabular} 
Tautologia
\item
\begin{tabular}{|c|c|c|c|}
\hline 
$p$ & $q$ & $(p\wedge q)$ & $((p\wedge q)\rightarrow p)$ \\ 
\hline 
0 & 0 & 0 & 1 \\ 
\hline 
0 & 1 & 0 & 1 \\ 
\hline 
1 & 0 & 0 & 1 \\ 
\hline 
1 & 1 & 1 & 1 \\ 
\hline 
\end{tabular} 
Tautologia
\item
\begin{tabular}{|c|c|c|c|c|c|}
\hline 
p & q & r & $(p\wedge (q\vee r))$ & $((p\wedge q)\vee (p\wedge r))$ & $((p\wedge (q\vee r))\leftrightarrow ((p\wedge q)\vee (p\wedge r)))$ \\ 
\hline 
0 & 0 & 0 & 0 & 0 & 1 \\ 
\hline 
0 & 0 & 1 & 0 & 0 & 1 \\ 
\hline 
0 & 1 & 0 & 0 & 0 & 1 \\ 
\hline 
0 & 1 & 1 & 0 & 0 & 1 \\ 
\hline 
1 & 0 & 0 & 0 & 0 & 1 \\ 
\hline 
1 & 0 & 1 & 1 & 1 & 1 \\ 
\hline 
1 & 1 & 0 & 1 & 1 & 1 \\ 
\hline 
1 & 1 & 1 & 1 & 1 & 1 \\  
\hline 
\end{tabular} 
Tautologia
\item
\begin{tabular}{|c|c|c|c|c|c|}
\hline 
p & q & r & $(p\rightarrow (q\rightarrow r))$ & $((p\rightarrow q)\rightarrow (p\rightarrow r))$ & $((p\rightarrow (q\rightarrow r))\leftrightarrow ((p\rightarrow q)\rightarrow (p\rightarrow r)))$ \\ 
\hline 
0 & 0 & 0 & 1 & 1 & 1 \\ 
\hline 
0 & 0 & 1 & 1 & 1 & 1 \\ 
\hline 
0 & 1 & 0 & 1 & 1 & 1 \\ 
\hline 
0 & 1 & 1 & 1 & 1 & 1 \\ 
\hline 
1 & 0 & 0 & 1 & 1 & 1 \\ 
\hline 
1 & 0 & 1 & 1 & 1 & 1 \\ 
\hline 
1 & 1 & 0 & 0 & 0 & 1 \\ 
\hline 
1 & 1 & 1 & 1 & 1 & 1 \\ 
\hline 
\end{tabular} 
Tautologia
\end{enumerate}

\paragraph{\textbf{Ejercicio 6.}} $\bigstar$ Dadas las proposiciones lógicas $\alpha$ y $\beta$,se dice que $\alpha$ es más fuerte que $\beta$ si y sólo si $\alpha \rightarrow \beta$ es una tautologia. En este caso, tambien decimos que $\beta$ es más debil que $\alpha$. Determinar la relacion de fuerza de los siguientes pares de fórmulas:

\begin{multicols}{2}
\begin{enumerate}[label=\alph*)]
\item $ True,False$
\item $ (p \wedge q),(p \vee q) $
\item $ True,True$
\item $ p,(p \wedge q)$
\item $ False,False $ 
\item $ p,(p \vee q) $
\item $ p,q$
\item $ p,(p \rightarrow q)$          
\end{enumerate}
\end{multicols}

¿Cuál es la proposición más fuerte y cuál la más debil de las que aparecen en este ejercicio?

\subsection*{Respuesta}

\begin{multicols}{2}
\begin{enumerate}[label=\alph*)]
\item $ False$
\item $ (p \wedge q)$
\item $ True$
\item $ (p \wedge q)$
\item $ False $ 
\item $ p $
\item $ Inmesurable$
\item Ninguna es tautologica, Inmesurable               
\end{enumerate}
\end{multicols}

La más fuerte es False, y la más debil es True, y ya me olvide el porque.

\paragraph{
\textbf{Ejercicio 7.} }$\bigstar$ Usando reglas de equivalencia (conmutatividad, asociatividad, De Morgan, etc) determinar si los siguientes pares de fórmulas son equivalencias.Indicar en cada paso qué regla se utilizó


\begin{enumerate}[label=\alph*)]
\item \begin{itemize}
\item $ ((\neg p\vee \neg q)\vee (p\wedge q))\rightarrow (p\wedge q)$
\item $ (p\wedge q)$
\end{itemize}
\item \begin{itemize}
\item $(p\vee q)\wedge (p\vee r)$
\item $ \neg p \rightarrow (q\wedge r)$
\end{itemize}
\item \begin{itemize}
\item $ \neg (\neg p)\rightarrow (\neg (\neg p \wedge \neg q))$
\item $ q $
\end{itemize}
\item \begin{itemize}
\item $((True \wedge p) \wedge (\neg p \vee False))\rightarrow \neg (\neg p\vee q)$
\item $ p \wedge \neg q$
\end{itemize}
\item \begin{itemize}
\item $ (p \vee (\neg p \wedge q))$
\item $ \neg p \rightarrow q$
\end{itemize}
\item \begin{itemize}
\item $ \neg (p \wedge (q \wedge s))$
\item $ s \rightarrow (\neg p \vee \neg q)$
\end{itemize}
\item \begin{itemize}
\item $ p \rightarrow (q \wedge \neg (q \rightarrow r))$
\item $ (\neg p \vee q)\wedge (\neg p \vee (q \wedge \neg r))$
\end{itemize}
\end{enumerate}

\subsection*{Respuesta}

\begin{enumerate}[label=\alph*)]
\item \begin{displaymath}
((\neg p\vee \neg q)\vee (p\wedge q))\rightarrow (p\wedge q) \qquad \textrm{(De Morgan)}
\end{displaymath}
\begin{displaymath}
( \neg (p \wedge q) \vee (p\wedge q))\rightarrow (p\wedge q) \qquad (\neg (p \wedge q) \vee (p\wedge q))=True \qquad \forall (p,q)
\end{displaymath}
\begin{displaymath}
True \rightarrow (p\wedge q) \qquad (True \rightarrow p )=p
\end{displaymath}
\begin{displaymath}
(p\wedge q)
\end{displaymath}
\item \begin{align*}
\neg p \rightarrow (q\wedge r) & \qquad & (a \rightarrow b)=(\neg a \vee b)\\
(\neg (\neg p) \vee (q\wedge r) & \qquad & \neg (\neg a)=a \\
(p \vee (q\wedge r)) & \qquad & \textrm{Distributiva}\\
(p\vee q)\wedge (p\vee r) &
\end{align*}
\item \begin{align*}
\neg (\neg p)\rightarrow (\neg (\neg p \wedge \neg q)) & \qquad & \neg (\neg a)=a\\
 p\rightarrow (\neg (\neg p \wedge \neg q)) & \qquad & \textrm{De Morgan}\\
 p\rightarrow (\neg (\neg( p \vee q))) & \qquad & \neg (\neg a)=a\\
 p\rightarrow ( p \vee q) & \qquad & (a \rightarrow b)=(\neg a \vee b)\\
 (\neg p \vee ( p \vee q)) & \qquad & (\neg a \vee a)=True\\
 q
\end{align*}
\item \begin{align*}
((True \wedge p) \wedge (\neg p \vee False))\rightarrow \neg (\neg p\vee q) & \qquad & (True \wedge a)=a\\
(p \wedge (\neg p \vee False))\rightarrow \neg (\neg p\vee q) & \qquad & (False \vee a)=a\\
(p \wedge \neg p)\rightarrow \neg (\neg p\vee q) & \qquad & (a \wedge \neg a)=False\\
False\rightarrow \neg (\neg p\vee q) & \qquad & (False \rightarrow a)=a\\
\neg (\neg p\vee q) & \qquad & Distributiva\\
p \wedge \neg q
\end{align*}
\item \begin{align*}
 \neg p \rightarrow q & \qquad & (a \rightarrow b)=(\neg a \vee b)\\
 (p \vee q) \wedge True  & \qquad & True=(a \vee \neg a)\\
 (p \vee q) \wedge (p \vee \neg p)  & \qquad & Distributiva\\
 (p \vee (\neg p \wedge q)) 
\end{align*}
\item \begin{align*}
 s \rightarrow (\neg p \vee \neg q) & \qquad & (a \rightarrow b)=(\neg a \vee b)\\
 (\neg s) \vee (\neg p \vee \neg q) & \qquad & \textrm{De Morgan}\\
 (\neg s) \vee \neg(p \wedge q) & \qquad & \textrm{De Morgan}\\
 \neg ( s \wedge (p \wedge q) & \qquad & Conmutatividad\\ 
 \neg (p \wedge (q \wedge s)) 
\end{align*}
\item \begin{align*}
 p \rightarrow (q \wedge \neg (q \rightarrow r)) & \qquad & (a \rightarrow b)=(\neg a \vee b)\\
 \neg p \vee (q \wedge \neg (q \rightarrow r)) & \qquad & (a \rightarrow b)=(\neg a \vee b)\\
 \neg p \vee (q \wedge \neg (\neg q \vee r)) & \qquad & \textrm{De Morgan}\\
 \neg p \vee (q \wedge  (q \wedge \neg r)) & \qquad & Distributiva\\
 (\neg p \vee q)\wedge (\neg p \vee (q \wedge \neg r))
\end{align*}
\end{enumerate}

\paragraph{\textbf{Ejercicio 8.}} Decimos que un conectivo es expresable mediante otros si es posible escribir una fórmula utilizando exclusivamente estos últimos y que tenga la misma tabla de verdad que el primero (es decir, son equivalentes). Por ejemplo, la disyunción es expresable mediante la conjunción mas la negación, ya que $ (p \vee q)$ tiene la misma tabla de verdad que $ \neg ( \neg p \wedge \neg q) $.\\
Mostrar que cualquier fórmula de la lógica proposicional que utilice los conectivos $\neg$ (negación), $\wedge$ (conjunción), $\vee$ (disyunción), $\rightarrow$ (implicación), $\leftrightarrow$ (equivalencia) puede reescribirse utilizando sólo los conectivos $\neg$ y $\vee$. 


\subsection*{Respuesta}
\begin{align*}
 (a \wedge b) \qquad  &= \qquad & \neg (\neg a \vee \neg b) \\
 (a \rightarrow b) \qquad  &= \qquad &( \neg a \vee b)\\
 (a \leftrightarrow b) \qquad &= \qquad &\neg ((a \vee b) \vee \neg (\neg (\neg a \vee \neg b)))
\end{align*}

\paragraph{\textbf{Ejercicio 9.}} $\bigstar$ Sean las variables proposicionales $f$, $e$ y $m$ con los siguientes significados:
\begin{multicols}{3}
\begin{itemize}
\item $ f \equiv$ "es fin de semana"
\item $ e \equiv$ "Juan estudia"
\item $ m \equiv$ "Juan escucha música"
\end{itemize}
\end{multicols}

\begin{enumerate}[label=\alph*)]
\item Escribir usando lógica proposicional las siguientes oraciones:\\
\begin{itemize}
\item[•] "Si es fin de semana, Juan estudia o escucha música, pero no ambas cosas"
\item[•] "Si no es fin de semana entonces Juan no estudia"
\item[•] "Cuando Juan estudia los fines de semana, lo hace escuchando música"
\end{itemize}
\item Asumiendo que valen las tres proposiciones anteriores ¿se puede deducir que Juan no estudia?. Justificar usando argumentos de la lógica proposicional.
\end{enumerate}

\subsection*{Respuestas}

\begin{enumerate}[label=\alph*)]
\item 
\begin{itemize}
\item[•] $f \rightarrow (\neg (e \leftrightarrow m)) $
\item[•] $ \neg f \rightarrow \neg e$
\item[•] $ (f\wedge e) \rightarrow m$
\end{itemize}
\item \begin{itemize}
\item Simplificaciones:
\begin{align*}
f \rightarrow (\neg (e \leftrightarrow m)) &= \neg f \vee (\neg (e \leftrightarrow m))\\
&= \neg f \vee ((e \vee m)\wedge \neg (e\wedge m))\\
\\
\neg f \rightarrow \neg e &= f \vee \neg e \\
\\
(f\wedge e) \rightarrow m &= \neg (f\wedge e)  \vee m \\ &=(\neg f\vee \neg e\vee m)\\
\end{align*}
\end{itemize}
Demostración:
\begin{align*}
((\neg f \vee ((e \vee m)\wedge \neg (e\wedge m)))\wedge (f \vee \neg e)\wedge (\neg (f\wedge e)  \vee m))\rightarrow \neg e &=\\
\neg ((\neg f \vee ((e \vee m)\wedge \neg (e\wedge m)))\wedge (f \vee \neg e)\wedge (\neg (f\wedge e)  \vee m))\vee \neg e &=?\\
\end{align*}
\end{enumerate}
\begin{center}
\begin{tabular}{|c|c|c|c|c|c|c|}
\hline 
$e$ & $f$ & $m$ & $A=(\neg f \vee ((e \vee m)\wedge \neg (e \wedge m)))$ & $B=(f\vee \neg e)$ & $C=(\neg f\vee \neg e\vee m)$ & $\neg(A\wedge B\wedge C)\vee \neg e$ \\ 
\hline 
0 & 0 & 0 & 1 & 1 & 1 & 1 \\ 
\hline 
0 & 0 & 1 & 1 & 1 & 1 & 1 \\ 
\hline 
0 & 1 & 0 & 0 & 1 & 1 & 1 \\ 
\hline 
0 & 1 & 1 & 1 & 1 & 1 & 1 \\ 
\hline 
1 & 0 & 0 & 1 & 0 & 1 & 1 \\ 
\hline 
1 & 0 & 1 & 1 & 0 & 1 & 1 \\ 
\hline 
1 & 1 & 0 & 1 & 1 & 0 & 1 \\ 
\hline 
1 & 1 & 1 & 0 & 1 & 1 & 1 \\ 
\hline 
\end{tabular} 
\end{center}

\paragraph{\textbf{Ejercicio 10.}} En la salita verde de un jardin se sabe que las siguientes circunstancias son ciertas:
\begin{enumerate}[label=\alph*)]
\item Si todos conocen a Juan entonces todos conocen a Camila (podemos pensar que esto se debe a que siempre caminan juntos).
\item Si todos conocen a Juan, entonces que todos conozcan a Camila implica que todos conocen a Gonzalo.\\
\end{enumerate}
La pregunta entonces es: ¿Es cierto que si todos conocen a Juan entonces todos conocen a Gonzalo? Justificar.

\subsection*{Respuesta}
$A:$ todos conocen a Juan, $B:$todos conocen a Camila, $C:$ todos conocen a Gonzalo
\begin{enumerate}[label=\alph*)]
\item $A\rightarrow B$
\item $A\rightarrow (B\rightarrow C)$
\end{enumerate}
\begin{center}
\begin{tabular}{|c|c|c|c|c|c|}
\hline 
$A$ & $B$ & $C$ & $A\rightarrow B$ & $A\rightarrow (B\rightarrow C)$ & $((A\rightarrow B)\wedge (A\rightarrow (B\rightarrow C))\rightarrow C$ \\
\hline 
0 & 0 & 0 & 1 & 1 & 1 \\ 
\hline 
0 & 0 & 1 & 1 & 1 & 1 \\ 
\hline 
0 & 1 & 0 & 1 & 1 & 1 \\ 
\hline 
0 & 1 & 1 & 1 & 1 & 1 \\ 
\hline 
1 & 0 & 0 & 0 & 1 & 1 \\ 
\hline 
1 & 0 & 1 & 0 & 1 & 1 \\ 
\hline 
1 & 1 & 0 & 1 & 0 & 0 \\ 
\hline 
1 & 1 & 1 & 1 & 1 & 1 \\ 
\hline 
\end{tabular} 
\end{center}

\paragraph{\textbf{Ejercicio 11.}} Siempre que Haroldo se pelea con sus compañeritos, vuelve a casa con un ojo morado. Si un dia lo viéramos llegar con el ojo destrozado, podríamos sentirnos inclinados a concluir que se ha tomado a golpes de puño y cabezazos con los otros niñitos. ¿Puede identificar el error en el razonamiento anterior? Pista: Es conocido como falacio de afirmar el consecuente

\subsection*{Respuesta}

$a\rightarrow b$ no implica que $b\rightarrow a$, ver T.V.

\section{Lógica ternaria (Verdadero, Falso o Indefinido)}

\paragraph{\textbf{Ejercicio 12.}} $\bigstar$ Asignar un valor de verdad (verdadero, falso o indefinido) a cada una de las siguientes expresiones ariméticas en los reales.

\begin{multicols}{3}
\begin{enumerate}[label=\alph*)]
\item $5>0$
\item $1\leq 1$
\item $(5+3-8)^{-1}\neq 2$
\item $0 \geq 5$
\item $\frac{1}{0} = \frac{1}{0}$
\item $0 > log_2(2^{2^0-1}-1)$
\item $0\cdot \sqrt{-1}=0$
\item $\sqrt{-1} \cdot 0=0$
\item $ \tan (\frac{\pi}{2})=\tan (\pi) - \tan (2)$
\end{enumerate}
\end{multicols}

\subsection*{Respuestas}

\begin{multicols}{3}
\begin{enumerate}[label=\alph*)]
\item $True$
\item $True$
\item $Undefined$
\item $False$
\item $Undefined$
\item $Undefined$
\item $True$
\item $Undefined$
\item $Undefined$
\end{enumerate}
\end{multicols}

\paragraph{\textbf{Ejercicio 13.}} $\bigstar$ ¿Cuál es la diferencia entre el operador $\wedge $ y el operador $\wedge _{L} $?Describir la tabla de verdad de ambos operadores. 

\subsection*{Respuesta}
\begin{center}
\begin{tabular}{|c|c|c|c|}
\hline 
$a$ & $b$ & $a\wedge b$ & $a\wedge _L b$ \\ 
\hline 
0 & 0 & 0 & 0 \\ 
\hline 
0 & 1 & 0 & 0 \\ 
\hline 
1 & 0 & 0 & 0 \\ 
\hline 
1 & 1 & 1 & 1 \\ 
\hline 
0 & $ \perp$ & $ \perp$ & 0 \\ 
\hline 
1 & $ \perp$ & $ \perp$ & $ \perp$ \\ 
\hline 
$ \perp$ & 0 & $ \perp$ & $ \perp$ \\ 
\hline 
$ \perp$ & 1 & $ \perp$ & $ \perp$ \\ 
\hline 
$ \perp$ & $ \perp$ & $ \perp$ & $ \perp$ \\ 
\hline 
\end{tabular} 
\end{center}

\paragraph{\textbf{Ejercicio 14.}} $\bigstar$ ¿Cuál es la diferencia entre el operador $\vee $ y el operador $\vee _{L} $?Describir la tabla de verdad de ambos operadores.

\subsection*{Respuesta}
\begin{center}
\begin{tabular}{|c|c|c|c|}
\hline 
$a$ & $b$ & $a\vee b$ & $a\vee _L b$ \\ 
\hline 
0 & 0 & 0 & 0 \\ 
\hline 
0 & 1 & 1 & 1 \\ 
\hline 
1 & 0 & 1 & 1 \\ 
\hline 
1 & 1 & 1 & 1 \\ 
\hline 
0 & $ \perp$ & $ \perp$ & $\perp$ \\ 
\hline 
1 & $ \perp$ & $ \perp$ &  1 \\ 
\hline 
$ \perp$ & 0 & $ \perp$ & $ \perp$ \\ 
\hline 
$ \perp$ & 1 & $ \perp$ & $ \perp$ \\ 
\hline 
$ \perp$ & $ \perp$ & $ \perp$ & $ \perp$ \\ 
\hline 
\end{tabular} 
\end{center}


\paragraph{\textbf{Ejercicio 15.}} $\bigstar$ ¿Cuál es la diferencia entre el operador $\rightarrow $ y el operador $\rightarrow _{L} $?Describir la tabla de verdad de ambos operadores.

\subsection*{Respuesta}
\begin{center}
\begin{tabular}{|c|c|c|c|}
\hline 
$a$ & $b$ & $a\rightarrow b$ & $a\rightarrow _L b$ \\ 
\hline 
0 & 0 & 1 & 1 \\ 
\hline 
0 & 1 & 1 & 1 \\ 
\hline 
1 & 0 & 0 & 0 \\ 
\hline 
1 & 1 & 1 & 1 \\ 
\hline 
0 & $ \perp$ & $ \perp$ & 1 \\ 
\hline 
1 & $ \perp$ & $ \perp$ &  $\perp$ \\ 
\hline 
$ \perp$ & 0 & $ \perp$ & $ \perp$ \\ 
\hline 
$ \perp$ & 1 & $ \perp$ & $ \perp$ \\ 
\hline 
$ \perp$ & $ \perp$ & $ \perp$ & $ \perp$ \\ 
\hline 
\end{tabular} 
\end{center}

\paragraph{\textbf{Ejercicio 16.}} $\bigstar$ Determinar los valores de verdad de las siguientes proposiciones cuando el valor de verdad de b y c es verdadero, el de a es falso y el de x e y es indefinido 
\begin{multicols}{2}
\begin{enumerate}[label=\alph*)]
\item $( \neg x \vee _L b)$
\item $ ((c \vee _L (y \wedge _L a))\vee b) $
\item $ \neg (c \vee _L y) $
\item $ (\neg (c\vee _L y)\leftrightarrow (\neg c \wedge _L \neg y)) $
\item $ ((c \vee _L y)\wedge _L (a\vee b)) $
\item $(((c\vee _L y)\wedge _L (a \vee _L b))\leftrightarrow _L (c\vee _L (y \wedge _L a)\vee _Lb))$
\item $(\neg c \wedge _L \neg y)$
\end{enumerate}
\end{multicols}

\subsection*{Respuestas}

\begin{multicols}{2}
\begin{enumerate}[label=\alph*)]
\item $Undefined$
\item $True$
\item $False$
\item $True$
\item $True$
\item $True?$
\item $False$
\end{enumerate}
\end{multicols}

\paragraph{
\textbf{Ejercicio 17.}}Sean $p$, $q$ y $r$ tres variables de las que se sabe que:

\begin{enumerate}
\item[•] p y q nunca están indefinidas,
\item[•] r se indefine si q es verdadera
\end{enumerate}
Proponer una fórmula que nunca se indefina, utilizando siempre las tres variables y que sea verdadera si y solo si se cumple que:
\begin{multicols}{2}
\begin{enumerate}[label=\alph*)]
\item Al menos una es verdadera
\item Ninguna es verdadera
\item Exactamente una de las tres es verdadera
\item Sólo p y q son verdaderas
\item No todas al mismo tiempo son verdaderas
\item r es verdadera
\end{enumerate}
\end{multicols}


\subsection*{Respuestas}

\begin{multicols}{2}
\begin{enumerate}[label=\alph*)]
\item $ (p\vee q)\vee _L r$
\item $ \neg ((p\vee q)\vee _L r)$
\item $ (p\wedge \neg q \wedge \neg r)\vee (\neg p\wedge q \wedge \neg r)\vee (\neg p\wedge \neg q \wedge r) $
\item $ (p \wedge q)\wedge _L \neg r$
\item $ \neg ((p\wedge q)\wedge (p\vee _L q))$
\item $ \neg q \wedge _L r$
\end{enumerate}
\end{multicols}

\section{Cuantificadores}

\paragraph{\textbf{Ejercicio 18.}}
\begin{enumerate}[label=\alph*)]
\item $\bigstar$ Determinar para cada aparición de variables, si dicha aparición se encuentra libre o ligada. En caso de estar ligada, aclarar a qué cuantificador lo está.
\begin{itemize}
\item[I)] $( \forall x:\mathbb{Z})(0\leq x < n \rightarrow x+y=z)$
\item[II)] $( \forall x:\mathbb{Z})( \forall y:\mathbb{Z})(0\leq x < n \wedge 0\leq y <m)\rightarrow x+y=z)) $
\item[III)] $( \forall j:\mathbb{Z})(0 \leq j < 10 \rightarrow j < 0) $
\item[IV)] $ s\wedge a < b-1 \wedge (( \forall j:\mathbb{Z})(a \leq j < b \rightarrow _L 2*j<b\vee s))$
\item[V)] $ ( \forall j:\mathbb{Z})(j \leq 0 \rightarrow ( \forall j:\mathbb{Z})(j>0 \rightarrow j \neq 0))$
\item[VI)] $ ( \forall j:\mathbb{Z})(j \leq 0 \rightarrow P(j))$
\item[VII)] $ ( \forall j:\mathbb{Z})(j \leq 0 \rightarrow P(j))\wedge P(j)$
\end{itemize}
\item $\bigstar$ En los casos en que sea posible, proponer valores para las variables libres del item anterior de modo tal que las expresiones sean verdaderas.
\end{enumerate}

\subsection*{Respuestas}

\begin{enumerate}[label=\alph*)]
\item
\begin{itemize}
\item[I)] $ x$ ligada a $\forall$; $n,y,z$ estan libres.
\item[II)] $ x,y$ ligadas a $\forall$; $n,m,z$ estan libres.
\item[III)] $ j$ ligada a $\forall$.
\item[IV)] $ j$ ligada a $\forall$; $a,b,s$ estan libres.
\item[V)]  $ j,j$ ligadas a $\forall$; (son jotas distintas, yo no escribi el ejercicio...).
\item[VI)] $ j$ ligada a $\forall$.
\item[VII)] $ j$ ligada a $\forall$; la otra $j$ esta libre.
\end{itemize}
\item
\begin{itemize}
\item[I)] $n=20;y=-x;z=0$
\item[II)] $n=1;m=1;z=0$
\item[III)] $\nexists$ ninguna variable libre.
\item[IV)] nah, muy denso...
\item[V)] $\nexists$ ninguna variable libre.
\item[VI)] mmm...
\item[VII)] mmm...
\end{itemize}
\end{enumerate}

\paragraph{\textbf{Ejercicio 19.}} Sean $P(x: \mathbb{Z})$ y $Q(x: \mathbb{Z})$ dos predicados cualesquiera que nunca se indefinen. Considerar los siguientes enunciados y su predicado asociado. Determinar, en cada caso, por qué el predicado no refleja correctamente el enunciado.
Corregir los errores.
\begin{enumerate}[label=\alph*)]
\item \textit{"Todos los naturales menores a 10 que cumplen P, cumplen Q":}\\
 pred $a()\{(\forall x: \mathbb{Z})((0 \leq x < 10)\rightarrow _L(P(x)\wedge Q(x)))\}$
\item \textit{"No hay ningún natural menor a 10 que cumpla P y Q":}\\
 pred $c()\{\neg ((\exists x: \mathbb{Z} )(0 \leq x <10 \wedge P(x)\wedge \neg ((\exists x: \mathbb{Z} )(0 \leq x <10 \wedge Q(x))))\}$
\end{enumerate}

\subsection*{Respuestas}
\begin{enumerate}[label=\alph*)]
\item $pred$ $a()\{(\forall x: \mathbb{Z})(((0 \leq x < 10)\wedge _L P(x))\rightarrow Q(x))\}$
\item $pred$ $c()\{\neg (\exists x: \mathbb{Z})((0 \leq x < 10)\wedge _L(P(x)\wedge Q(x)))\}$
\end{enumerate}
\section{Funciones auxiliares}

\paragraph{\textbf{Ejercicio 20.}} $\bigstar$ Escriba los siguientes predicados y funciones en el lenguaje de especificación:
\begin{enumerate}[label=\alph*)]
\item \textit{aux suc} $(x: \mathbb{Z}):\mathbb{Z}$ , que corresponde al sucesor de $x$.
\item \textit{aux suma} $(x,y: \mathbb{R}):\mathbb{R}$ , que corresponda a la suma entre $x$ e $y$.
\item \textit{aux producto} $(x,y: \mathbb{R}):\mathbb{R}$ , que corresponda al producto entre $x$ e $y$.
\item \textit{pred esCuadrado} $(x: \mathbb{Z})$ , que sea verdadero si y solo si $x$ es un numero cuadrado.
\item \textit{pred esPrimo} $(x: \mathbb{Z})$ , que sea verdadero sii $x$ es primo.
\item \textit{pred sonCoprimos} $(x,y: \mathbb{Z})$ , que sea verdadero si y solo si $x$ e $y$ son coprimos.
\item \textit{pred divisoresGrandes} $(x,y: \mathbb{Z})$ , que sea verdadero cuando todos los divisores de $x$, sin contar el uno, son mayores que $y$.
\item \textit{pred mayorPrimoQueDivide} $(x: \mathbb{Z},y: \mathbb{Z})$ , que sea verdadero si $y$ es el mayor primo que divide a $x$.
\item \textit{pred sonPrimosHermanos} $(x: \mathbb{Z},y: \mathbb{Z})$ , que sea verdadero cuando $x$ es primo, $y$ es primo, y son primos consecutivos.
\end{enumerate}

\subsection*{Respuestas}

\begin{enumerate}[label=\alph*)]
\item \textit{aux suc} $(x: \mathbb{Z}):\mathbb{Z}=x+1;$ 
\item \textit{aux suma} $(x,y: \mathbb{R}):\mathbb{R}=x+y;$
\item \textit{aux producto} $(x,y: \mathbb{R}):\mathbb{R}=x*y;$
\item \textit{pred esCuadrado} $(x: \mathbb{Z})\{$\\ 
\hspace*{6mm}$(\exists a :\mathbb{Z}_0)(a*a=x)$\\
$\}$
\item \textit{pred esPrimo} $(x: \mathbb{Z})\{$\\ 
\hspace*{6mm}$ (x > 1)(\forall i :\mathbb{Z})(1<i<x \rightarrow _L x\textrm{ mod }i\neq 0 )$\\
$\}$
\item \textit{pred sonCoprimos} $(x,y: \mathbb{Z})\{$\\  
\hspace*{6mm}$ \neg (\exists i :\mathbb{Z})(1<i<max\{abs(x),abs(y)\} \wedge (x\textrm{ mod }i=0)\wedge (y\textrm{ mod }i=0))$\\
$\}$
\item \textit{pred divisoresGrandes} $(x,y: \mathbb{Z})\{$\\
\hspace*{6mm}$ (\forall i :\mathbb{Z})(1<i<abs(x) \wedge _L(x\textrm{ mod }i=0 ) \rightarrow  i>y)$\\
$\}$
%\item \textit{pred mayorPrimoQueDivide} $(x: \mathbb{Z},y: \mathbb{Z})\{$\\
%\hspace*{6mm}$esPrimo(y)\wedge _L (x $ mod $y=0)\wedge$\\
%\hspace*{6mm}$(\forall n' :\mathbb{Z})((n'>abs(y) \wedge _L (x $ mod $n' =0))  \rightarrow  \neg esPrimo(n') )$\\
%$\}$

\item \textbf{pred esPrimo} $(n: \mathbb{Z})\{$\\ 
			\hspace*{6mm}$ n > 1 \wedge(\forall i :\mathbb{Z})
			(1<i<n \rightarrow _L n\textrm{ mod }i\neq 0 )$\\
			\hspace*{5mm}$\}$\smallskip \\
\item
\end{enumerate}
\end{document}