% !TEX TS-program = pdflatex
% !TEX encoding = UTF-8 Unicode

% This is a simple template for a LaTeX document using the "article" class.
% See "book", "report", "letter" for other types of document.

% align equations to the left
% use larger type; default would be 10pt
%\documentclass[fleqn, 11pt]{article}
\documentclass[a4paper]{article}
\usepackage{a4wide}
%\usepackage{amsmath, amscd}
\usepackage[spanish,activeacute]{babel}
\usepackage{enumerate}
\usepackage{xspace}
\usepackage{longtable}
\usepackage{graphics}
\usepackage{listings}
%\usepackage[utf8]{inputenc} % set input encoding (not needed with XeLaTeX)

%%% Examples of Article customizations
% These packages are optional, depending whether you want the features they
% provide.
% See the LaTeX Companion or other references for full information.

%%% PAGE DIMENSIONS
%\usepackage{geometry} % to change the page dimensions
%\geometry{a4paper} % or letterpaper (US) or a5paper or....
% for example, change the margins to 2 inches all round
%\geometry{margin=12mm}
% set up the page for landscape
% \geometry{landscape}
%   read geometry.pdf for detailed page layout information

% support the \includegraphics command and options
%\usepackage{graphicx}

% Activate to begin paragraphs with an empty line rather than an indent
%\usepackage[parfill]{parskip}

%%% PACKAGES
% for much better looking tables
%\usepackage{booktabs}
% for better arrays (eg matrices) in maths
%\usepackage{array}
% very flexible & customisable lists (eg. enumerate/itemize, etc.)
%\usepackage{paralist}
% adds environment for commenting out blocks of text & for better verbatim
%\usepackage{verbatim}
% make it possible to include more than one captioned figure/table in a single
% float
%\usepackage{subfig}
% These packages are all incorporated in the memoir class to one degree or
% another...

%%% HEADERS & FOOTERS
%\usepackage{fancyhdr} % This should be set AFTER setting up the page geometry
%\pagestyle{fancy} % options: empty , plain , fancy
%\renewcommand{\headrulewidth}{0pt} % customise the layout...
%\lhead{}\chead{}\rhead{}
%\lfoot{}\cfoot{\thepage}\rfoot{}

%%% SECTION TITLE APPEARANCE
%\usepackage{sectsty}
% (See the fntguide.pdf for font help)
%\allsectionsfont{\sffamily\mdseries\upshape}
% (This matches ConTeXt defaults)

%%% ToC (table of contents) APPEARANCE
% Put the bibliography in the ToC
%\usepackage[nottoc, notlof, notlot]{tocbibind}
% Alter the style of the Table of Contents
%\usepackage[titles, subfigure]{tocloft}
%\renewcommand{\cftsecfont}{\rmfamily\mdseries\upshape}
%\renewcommand{\cftsecpagefont}{\rmfamily\mdseries\upshape} % No bold!

\usepackage{calc}
\usepackage{lmodern}
\usepackage{amssymb}
\usepackage{amsmath}
\usepackage{mathdots}
\usepackage{mathtools}
\usepackage{multicol}
\usepackage{enumitem}
\usepackage{tasks}

\input{../../macros/Algo1Macros}

%\newcommand{\nat}{\mathbb{N}}
%\newcommand{\Ccur}{\mathcal{C}}
%\newcommand{\indef}{\uparrow}

%\overfullrule=2em

%%% END Article customizations

%%% The "real" document content comes below...

\title{Resuelto Entregable Algoritmos y Estructuras de Datos I}
\author{Andres M. Hense,Victoria Espil}
\date{} % Activate to display a given date or no date (if empty),
% otherwise the current date is printed 

\begin{document}
%\maketitle


%\section*{Práctica 1 --- Lógica}

\practica{1}{Ejercicios Entregables Resueltos}

\begin{center}
\textbf{Integrantes:} Andrés M. Hense, Victoria Espil
\end{center}
\paragraph{\textbf{Ejercicio 5.h}} Determinar, utilizando tablas de verdad, si las siguientes fórmulas son tautologias, contradicciones o contingencias.

\begin{itemize}
\item $ ((p \wedge (q \vee r)) \leftrightarrow ((p \wedge q)\vee (p \wedge r)))$               
\end{itemize}

\subsection*{Respuesta}

\begin{tabular}{|c|c|c|c|c|c|}
\hline 
p & q & r & $(p\wedge (q\vee r))$ & $((p\wedge q)\vee (p\wedge r))$ & $((p\wedge (q\vee r))\leftrightarrow ((p\wedge q)\vee (p\wedge r)))$ \\ 
\hline 
0 & 0 & 0 & 0 & 0 & 1 \\ 
\hline 
0 & 0 & 1 & 0 & 0 & 1 \\ 
\hline 
0 & 1 & 0 & 0 & 0 & 1 \\ 
\hline 
0 & 1 & 1 & 0 & 0 & 1 \\ 
\hline 
1 & 0 & 0 & 0 & 0 & 1 \\ 
\hline 
1 & 0 & 1 & 1 & 1 & 1 \\ 
\hline 
1 & 1 & 0 & 1 & 1 & 1 \\ 
\hline 
1 & 1 & 1 & 1 & 1 & 1 \\  
\hline 
\end{tabular} 
Tautologia



\paragraph{\textbf{Ejercicio 16.d}} Determinar los valores de verdad de las siguientes proposiciones cuando el valor de verdad de b y c es verdadero, el de a es falso y el de x e y es indefinido 

\begin{itemize}
\item $ (\neg (c\vee _L y)\leftrightarrow (\neg c \wedge _L \neg y)) $
\end{itemize}


\subsection*{Respuesta}
\begin{align*}
  (\neg (c\vee _L y)\leftrightarrow (\neg c \wedge _L \neg y)) &= (\neg (True\vee _L \perp)\leftrightarrow (\neg True \wedge _L \neg \perp))\\
  &= (\neg True\leftrightarrow (False \wedge _L \neg \perp))\\
  &= (False\leftrightarrow False)\\
  &= True
\end{align*}



\paragraph{\textbf{Ejercicio 18.V}} Determinar para cada aparición de variables, si dicha aparición se encuentra libre o ligada. En caso de estar ligada, aclarar a qué cuantificador lo está.
\begin{itemize}
\item $ ( \forall j:\mathbb{Z})(j \leq 0 \rightarrow ( \forall j:\mathbb{Z})(j>0 \rightarrow j \neq 0))$
\end{itemize}


\subsection*{Respuesta}

$ ( \forall _1 j_{1}:\mathbb{Z})(j_{1_1} \leq 0 \rightarrow ( \forall _2 j_2:\mathbb{Z})(j_{2_1}>0 \rightarrow j_{2_2} \neq 0))$\\

$ j_1$ esta ligada a $\forall _1$, y es llamada en $j_{1_1}$

$ j_2$ esta ligada a $\forall _2$, y es llamada en $j_{2_1}$ y $j_{2_2}$



\paragraph{\textbf{Ejercicio 20.h}}  Escriba los siguientes predicados y funciones en el lenguaje de especificación:
\begin{itemize}
\item \textit{pred mayorPrimoQueDivide} $(x: \mathbb{Z},y: \mathbb{Z})$ , que sea verdadero si $y$ es el mayor primo que divide a $x$.
\end{itemize}

\subsection*{Respuesta}

\begin{itemize}
\item \textit{pred esPrimo} $(x: \mathbb{Z})\{$\\
\hspace*{6mm}$(x > 1)(\forall i :\mathbb{N})(1<i<x \rightarrow _L x\textrm{ mod }i\neq 0 )$\\
$\}$

%\item \textit{pred mayorPrimoQueDivide} $(x: \mathbb{Z},y: \mathbb{Z})\{$\\
%\hspace*{6mm}$esPrimo(y)\wedge _L (x $ mod $y=0)\wedge$\\
%\hspace*{6mm}$(\forall n' :\mathbb{Z})((n'>abs(y) \wedge _L (x $ mod $n' =0))  \rightarrow  \neg esPrimo(n') )$\\
%$\}$

\item \textit{pred mayorPrimoQueDivide} $(x: \mathbb{Z},y: \mathbb{Z})\{$\\
\hspace*{6mm}$esPrimo(y)\wedge _L (x $ mod $y=0)\wedge$\\
\hspace*{6mm}$(\forall i :\mathbb{N})((abs(y)<i<abs(x) \wedge _L(x $ mod $i =0)  )\rightarrow \neg esPrimo(i) )$\\
$\}$
\end{itemize}
\end{document}