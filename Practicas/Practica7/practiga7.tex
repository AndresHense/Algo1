
\documentclass{article}

\usepackage{a4wide}
\usepackage[utf8]{inputenc}
\usepackage{enumerate}
\usepackage{xspace}
\usepackage{longtable}
\usepackage{graphicx}
\usepackage{listings}
\usepackage{xcolor}
\usepackage{calc}
\usepackage{lmodern}
\usepackage{amssymb}
\usepackage{amsmath}
\usepackage{mathdots}
\usepackage{mathtools}
\usepackage{multicol}
\usepackage{enumitem}
\usepackage{tasks}
\usepackage{pdfpages}
\usepackage{listings}
\lstset{
	language=C++,
	numbers=left,
	tabsize=3,
	basicstyle=\ttfamily,
}

\usepackage{bera}
\lstset{basicstyle=\ttfamily,
escapeinside={||},
mathescape=true}

\input{../macros/Algo1Macros}

\begin{document}
\renewcommand{\thelstlisting}{\thesection.\arabic{lstlisting}}

\includepdf[pages=-]{practica7.pdf}

\begin{center}
\section*{RESOLUCIONES.}
\end{center}
\paragraph{Ejercicio 1}.

\begin{lstlisting}
vector<int> copiarSecuencia(vector<int> s){
	vector<int> r;
	int i=0;
	while(i<s.size()){
		r.push_back(s[i]);
		i++;	
	}
	return r;
}
\end{lstlisting}
.
\paragraph*{Ejercicio 2}.

\begin{lstlisting}
vector<int> incSecuencia(vector<int> &s){
	int i=0;
	while(i<s.size()){
		s[i]++;
		i++;	
	}
}
\end{lstlisting}
.
\paragraph*{Ejercicio 3}.

\begin{lstlisting}
int cantApariciones(vector<int> s, int e){
	int r=0;
	for(int i=0; i<s.size();i++){
		r=r+(s[i]==e);
	}
	return r;
}
\end{lstlisting}
.
\paragraph*{Ejercicio 4}.

\begin{lstlisting}
void raizConEsei(vector<int> &s){
	int i=-1;
	vector<int> s0=s;
	while( i<s.size()-1){
		i++;
		s[i]=sqrt(s0[i]);
	}
	return;
}
\end{lstlisting}
.
\paragraph*{Ejercicio 5}.

\begin{lstlisting}
void duplicarElementos(vector<int> &s){
	int i=0;
	vector<int> s0=s;
	while( i<s.size()/2){
		s[s.size()-2*i-1]=2*s0[s0.size()-2*i-1];
		s[s.size()-2*i-2]=2*s0[s0.size()-2*i-2];
		i++;
	}
	return;
}
\end{lstlisting}
.
\paragraph*{Ejercicio 6}.

\begin{lstlisting}
void dividirPorPromedio(vector<float> &s){
	int i=0;
	vector<float> s0=s;
	int p=promedio(s0);
	while( i<s.size()/2){
		s[i]=s0[i]/p;
		s[s.size()-i-1]=s0[s0.size()-i-1]/p;
		i++;
	}
	return;
}

float promedio(vector<float> &s){
	float res=0;
	for(int i=0;i<s.size();i++){
		res+=s[i];	
	}
	res/=s.size();
	return res;
}
\end{lstlisting}
\end{document}