
\documentclass[a4paper]{article}
\usepackage{a4wide}
%\usepackage{amsmath, amscd}
\usepackage[spanish,activeacute]{babel}
\usepackage{enumerate}
\usepackage{xspace}
\usepackage{longtable}
\usepackage{graphics}
\usepackage{listings}


\usepackage{calc}
\usepackage{lmodern}
\usepackage{amssymb}
\usepackage{amsmath}
\usepackage{mathdots}
\usepackage{mathtools}
\usepackage{multicol}
\usepackage{enumitem}
\usepackage{tasks}


\input{../../macros/Algo1Macros}


\title{Resuelto Entregable Algoritmos y Estructuras de Datos I}
\author{Andres M. Hense,Victoria Espil}
\date{} % Activate to display a given date or no date (if empty),
% otherwise the current date is printed 

\begin{document}
%\maketitle


%\section*{Práctica 1 --- Lógica}

\practica{3}{Ejercicios Entregables Resueltos}

\begin{center}
\textbf{Integrantes:} Andrés M. Hense, Victoria Espil
\end{center}

\paragraph*{Ejercicio 14.} Especificar los siguientes problemas:
	\begin{enumerate}[label=\alph*)]
		\item $\bigstar$ Dado un número entero positivo, obtener la suma de sus factores primos.
	\end{enumerate}

\paragraph*{Ejercicio 15.} Especificar los siguientes problemas sobre secuencias:
	\begin{enumerate}[label=\alph*)]
		\item $\bigstar$ Dadas dos secuencias $s$ y $t$, devolver su \textit{intersección}, es
				decir, una secuencia con todos los elemntos que aparecen en ambas. Si un mismo
				elemento tiene repetidos, la secuencia retornada debe contener la cantidad
				mínima de apariciones en de $s$ y de $t$.
	\end{enumerate}

\paragraph*{Ejercicio 22.} Especificar los siguientes problemas de modificación de secuencias:
	\begin{enumerate}[label=\alph*)]
		\item $\bigstar$ \textbf{proc primosHermanos}(inout $l:seq\langle \mathbb{Z}\rangle$), que dada una secuencia de enteros mayores a dos, reemplaza dichos valores por el número primo menor más cercano. Por ejemplo, si $l=\langle 6,5,9,14 \rangle$, luego de aplicar \textbf{primosHermanos}($l$), $l=\langle 5,5,7,13 \rangle$
	\end{enumerate}
\end{document}