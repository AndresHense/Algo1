\documentclass[a4paper]{article}
\usepackage{a4wide}
\usepackage[spanish,activeacute]{babel}
\usepackage{enumerate}
\usepackage{xspace}
\usepackage{longtable}
\usepackage{graphics}
\usepackage{listings}

\usepackage{calc}
\usepackage{lmodern}
\usepackage{amssymb}
\usepackage{amsmath}
\usepackage{mathdots}
\usepackage{mathtools}
\usepackage{multicol}
\usepackage{enumitem}
\usepackage{tasks}

\input{../macros/Algo1Macros}



\begin{document}

\practica{3}{Especificación De Problemas Resuelto}

\subsection*{Comentarios:}
Hola, este no es un resuelto oficial, tiene el logo del DC porque me parecio divertido copiar el formato de la guia.\\

\paragraph*{Ejercicio 1.}$\bigstar$ Las siguientes especificaciones no son correctas. Indicar por qué,y corregirlas para que describan correctamente el problema.
	\begin{enumerate}[label=\alph*)]
		\item \textbf{buscar:} Dada una secuencia y un elemento, devuelve en \textit{result} la posición de la secuencia en la cual se encuentra el elemento.\vspace{6mm} \\
		\textbf{proc buscar }(in l$:seq\langle \mathbb{R}\rangle$,in elem$:\mathbb{R}$,out result$:\mathbb{Z})\ \{$\smallskip \\
		\hspace*{6mm} \textbf{Pre }$\{ elem \in l\}$\smallskip \\
		\hspace*{6mm} \textbf{Post }$\{l[result]=elem\}$\\
		$\}$
		\item \textbf{progresionGeometricaFactor2:} Indica si la secuencia $l$ representa una progresión geométrica factor 2. Es decir, si cada elemento de la secuencia es el doble del elemento anterior.\vspace{6mm} \\
		\textbf{proc progresionGeometricaFactor2 }(in l$:seq\langle \mathbb{R}\rangle$,out result$:Bool\ \{$\smallskip \\
		\hspace*{6mm} \textbf{Pre }$\{ True\}$\smallskip \\
		\hspace*{6mm} \textbf{Post }$\{result=True\leftrightarrow ((\forall i:\mathbb{Z})(0\leq i < |l|\rightarrow_L l[i]=2*l[i-1]))\}$\\
		$\}$
		\item \textbf{minimo:} Devuelce en $result$ el menor elemento de $l$.\vspace{6mm} \\
		\textbf{proc minimo }(in l$:seq\langle \mathbb{R}\rangle$,out result$:\mathbb{Z})\ \{$\smallskip \\
		\hspace*{6mm} \textbf{Pre }$\{ True\}$\smallskip \\
		\hspace*{6mm} \textbf{Post }$\{(\forall y:\mathbb{Z})((y\in l \wedge y \neq x)\rightarrow y > result)\}$\\
		$\}$
	\end{enumerate}
\subsection*{Respuesta}
	\begin{enumerate}[label=\alph*)]
		\item La \textbf{Pre} no aclara que pasa cuando hay mas de una aparición de $elem$ en $l$, y hace falta pedir que $result$ este en el rango de $l$. \medskip \\
		\textbf{proc buscar }(in l$:seq\langle \mathbb{R}\rangle$,in elem$:\mathbb{R}$,out result$:\mathbb{Z})\ \{$\smallskip \\
		\hspace*{6mm} \textbf{Pre }$\{ elem \in l\wedge cantAparciones(elem,l)=1\}$\smallskip \\
		\hspace*{6mm} \textbf{Post }$\{0\leq result <|l| \wedge_L l[result]=elem\}$\\
		$\}$
		\item Este es más facil de ver que el anterior, cuando $i=0$, va a tratar de acceder a la posicion $l[0-1]$, que es cualquier cosa. Y creo que crashearia con una lista vacia o de un elemento. \medskip \\
		\textbf{proc progresionGeometricaFactor2 }(in l$:seq\langle \mathbb{R}\rangle$,out result$:Bool\ \{$\smallskip \\
		\hspace*{6mm} \textbf{Pre }$\{ True\}$\smallskip \\
		\hspace*{6mm} \textbf{Post }$\{result=True\leftrightarrow ((\forall i:\mathbb{Z})(0\leq i < |l|-1\rightarrow_L 2*l[i]=l[i+1]))\}$\\
		$\}$ 
		\item No se para que esta ese $y\neq x$, y tendria que haber pedido en la \textbf{Pre} que $result$ pertenezca a $l$.\medskip \\
		\textbf{proc minimo }(in l$:seq\langle \mathbb{R}\rangle$,out result$:\mathbb{Z})\ \{$\smallskip \\
		\hspace*{6mm} \textbf{Pre }$\{ result\in l\}$\smallskip \\
		\hspace*{6mm} \textbf{Post }$\{(\forall y:\mathbb{Z})(y\in l\rightarrow y > result)\}$\\
		$\}$
	\end{enumerate}
	
\paragraph*{Ejercicio 2.} La siguiente no es una especificación válida, ya que para ciertos valores de entrada que cumplen la precondición, no existe una salida que cumpla con la postcondición.\medskip \\
\textbf{proc elementosQueSumen }(in $l:seq\langle \mathbb{Z}\rangle,$in suma$:\mathbb{Z}$, out result $:seq\langle \mathbb{Z}\rangle)\ \{$\\
\hspace*{6mm}\textbf{Pre }$\{True\}$\\
\hspace*{6mm}\textbf{Post }$\{$\\
\hspace*{6mm}/* La secuencia result está incluída en la secuencia l */\\
\hspace*{6mm}$(\forall x:\mathbb{Z})(x\in $ result$ \rightarrow \#$apariciones($x,result)\leq \#$apariciones$(x,l)$)\\
\hspace*{6mm}/* La suma de la result coincide con el valor de la suma */\\
\hspace*{6mm}$\wedge suma=\sum_{i=0}^{|result|-1}result[i]$\\
\hspace*{6mm}$\}$\\
	\begin{enumerate}[label=\alph*)]
		\item Mostrar valores para $l$ y $suma$ que hagan verdadera la precondición, pero tales que no exista $result$ que cumpla la postcondición.
		\item Supongamos que agregamos a la especificación la siguiente cláusula:\smallskip \\
		\hspace*{4mm} \textbf{Pre} : $min\_suma(l)\leq suma\leq max\_suma(l)$\\
		\hspace*{4mm} \textbf{fun} $min\_suma(l):\mathbb{Z}=\sum_{i=0}^{|l|-1}$if $l[i]<0$ then $l[i]$ else 0 fi\\
		\hspace*{4mm} \textbf{fun} $max\_suma(l):\mathbb{Z}=\sum_{i=0}^{|l|-1}$if $l[i]>0$ then $l[i]$ else 0 fi\smallskip \\
		¿Ahora es una especificación válida? Si no lo es, justificarlo con un ejemplo como en el punto anterior.
		\item Dar una precondición que haga correcta la especificación
	\end{enumerate}
\subsection*{Respuesta}
	\begin{enumerate}[label=\alph*)]
		\item $l= \langle 9,9,9\rangle$, $suma=1$, si $l$ contiene a $result$, entonces necesariamente va a sumar por lo menos 9, por lo que no puede valer 1 su suma.
		\item $l= \langle 9,9,9\rangle$, $suma=1$, si $l$ contiene a $result$, entonces necesariamente va a sumar por lo menos 9, por lo que no puede valer 1 su suma, y ademas suma cumple la desigualdad $0\leq suma\leq 27$
		\item \textbf{proc elementosQueSumen }(in $l:seq\langle \mathbb{Z}\rangle,$in suma$:\mathbb{Z}$, out result $:seq\langle \mathbb{Z}\rangle)\ \{$\\
\hspace*{6mm}\textbf{Pre }$\{cantSubSeqCumplenSuma(l,suma)>0\}$\\
\hspace*{6mm}\textbf{Post }$\{$\\
\hspace*{6mm}/* La secuencia result está incluída en la secuencia l */\\
\hspace*{6mm}$(\forall x:\mathbb{Z})(x\in $ result$ \rightarrow \#$apariciones($x,result)\leq \#$apariciones$(x,l)$)\\
\hspace*{6mm}/* La suma de la result coincide con el valor de la suma */\\
\hspace*{6mm}$\wedge suma=\sum_{i=0}^{|result|-1}result[i]$\\
\hspace*{6mm}$\}$\medskip \\

		\textit{aux cantSubSeqCumplenSuma}$(l:seq\langle \mathbb{Z}\rangle,suma :\mathbb{Z}):\mathbb{Z}=$\smallskip \\
			\hspace*{4mm}$\sum_{j=1}^{|l|}\sum_{i=0}^{|l|-1}$ if $(|subseq(l,i,j)|>0 \wedge
			 sumaSeq(subseq(l,i,j))=suma)$ 
			then 1 else 0 fi \bigskip \\
		\textit{aux sumaSeq}$(l:seq\langle \mathbb{Z}\rangle):\mathbb{Z}=\sum_{k=0}^{|l|-1}l[k]$
	\end{enumerate}
	
\paragraph*{Ejercicio 3.}$\bigstar$ Para los siguientes problemas, dar todas las soluciones posibles a las entradas dadas.
	\begin{enumerate}[label=\alph*)]
		\item 
			\textbf{proc raizCuadrada }(in x$: \mathbb{R}$,out result$:\mathbb{R})\ \{$\smallskip \\
			\hspace*{6mm} \textbf{Pre }$\{ x\geq 0\}$\smallskip \\
			\hspace*{6mm} \textbf{Post }$\{result^2=x\}$\\
			$\}$
			\begin{enumerate}[label=\Roman*)]
				\item $x=0$
				\item $x=1$
				\item $x=27$
			\end{enumerate}
		\item $\bigstar$ \\
			\textbf{proc indiceDelMaximo }(in l$: seq\langle\mathbb{R}\rangle$,out result$:\mathbb{Z})\ \{$\smallskip \\
			\hspace*{6mm} \textbf{Pre }$\{ |l| > 0\}$\smallskip \\
			\hspace*{6mm} \textbf{Post }$\{$\\
			\hspace*{6mm} $0\leq result < |l|$\\
			\hspace*{6mm} $\wedge_L((\forall i :\mathbb{Z})(0 \leq i < |l|\rightarrow_L l[i]\leq l[result]))$\\
			\hspace*{6mm}$\}$\\
			$\}$
			\begin{enumerate}[label=\Roman*)]
				\item $l=\langle 1,2,3,4\rangle$
				\item $l=\langle 15.5,-18,4.215,15.5,-1\rangle$
				\item $l=\langle 0,0,0,0,0,0\rangle$
			\end{enumerate}
		\item
			$\bigstar$ \\
			\textbf{proc indiceDelPrimerMaximo }(in l$: seq\langle\mathbb{R}\rangle$,out result$:\mathbb{Z})\ \{$\smallskip \\
			\hspace*{6mm} \textbf{Pre }$\{ |l| > 0\}$\smallskip \\
			\hspace*{6mm} \textbf{Post }$\{$\\
			\hspace*{6mm} $0\leq result < |l|$\\
			\hspace*{6mm} $\wedge ((\forall i :\mathbb{Z})(0 \leq i < |l|\rightarrow_L (l[i] < l[result] \vee (l[i]=l[result]\wedge i \geq result))))$\\
			\hspace*{6mm}$\}$\\
			$\}$
			\begin{enumerate}[label=\Roman*)]
				\item $l=\langle 1,2,3,4\rangle$
				\item $l=\langle 15,5,-18,4.215,15.5,-1\rangle$
				\item $l=\langle 0,0,0,0,0,0\rangle$
			\end{enumerate}
		\item  ¿Para qué valores de entrada \textbf{indiceDelPrimerMaximo} y \textbf{indiceDelMaximo} tienen necesariamente la misma salida?
	\end{enumerate}
\subsection*{Respuesta}
	\begin{enumerate}[label=\alph*)]
		\item
			\begin{enumerate}[label=\Roman*)]
				\item $result=0$
				\item $result=1;-1$
				\item $result=3\sqrt{3};-3\sqrt{3}$
			\end{enumerate}
		\item
			\begin{enumerate}[label=\Roman*)]
				\item $result=3$
				\item Cualquier cosa, no dice nada cuando hay más de una aparición
						del maximo.
				\item Idem
			\end{enumerate}
		\item
			\begin{enumerate}[label=\Roman*)]
				\item $result=3$
				\item $result=0$, 
				\item $result=0$
			\end{enumerate}
		\item Van a tener la misma salida cuando no haya más de una aparición del maximo en la lista (ya que en caso contrario \textbf{indiceDelMaximo} crashearia).
	\end{enumerate}
	
\paragraph*{Ejercicio 4.}$\bigstar$ Sea $f:\mathbb{R}\times \mathbb{R}\rightarrow \mathbb{R}$ definida como:\\
\[ f(a,b)=\left\{ \begin{array}{ll}
			2b  & \textrm{si } a < 0\\
			b-1 & \textrm{en otro caso }
		\end{array}\right.
\]
¿Cuáles de las siguientes especificaciones son correctas para el problema de calcular $f(x,y)$?\\
Para las que no lo son, indicar por qué.\bigskip \\
	\begin{enumerate}[label=\alph*)]
		\item
			\textbf{proc f }(in a, b$: \mathbb{R}$,out result$:\mathbb{R})\ \{$\smallskip \\
			\hspace*{6mm} \textbf{Pre }$\{ True\}$\smallskip \\
			\hspace*{6mm} \textbf{Post }$\{$\\
			\hspace*{6mm} $(a < 0 \wedge result=2*b)$\\
			\hspace*{6mm} $\wedge$\\
			\hspace*{6mm} $(a \geq 0 \wedge result=b-1)$\\			
			\hspace*{6mm}$\}$\\
			$\}$
		\item
			\textbf{proc f }(in a, b$: \mathbb{R}$,out result$:\mathbb{R})\ \{$\smallskip \\
			\hspace*{6mm} \textbf{Pre }$\{ True\}$\smallskip \\
			\hspace*{6mm} \textbf{Post }$\{(a<0 \wedge result=2*b)\vee (a> 0 \wedge result=b-1)\}$\\
			$\}$
		\item
			\textbf{proc f }(in a, b$: \mathbb{R}$,out result$:\mathbb{R})\ \{$\smallskip \\
			\hspace*{6mm} \textbf{Pre }$\{ True\}$\smallskip \\
			\hspace*{6mm} \textbf{Post }$\{(a<0 \wedge result=2*b)\vee (a\geq 0 \wedge result=b-1)\}$\\
			$\}$
		\item
			\textbf{proc f }(in a, b$: \mathbb{R}$,out result$:\mathbb{R})\ \{$\smallskip \\
			\hspace*{6mm} \textbf{Pre }$\{ True\}$\smallskip \\
			\hspace*{6mm} \textbf{Post }$\{$\\
			\hspace*{6mm} $(a < 0 \rightarrow result=2*b)$\\
			\hspace*{6mm} $\wedge$\\
			\hspace*{6mm} $(a \geq 0 \rightarrow result=b-1)$\\			
			\hspace*{6mm}$\}$\\
			$\}$
		\item
			\textbf{proc f }(in a, b$: \mathbb{R}$,out result$:\mathbb{R})\ \{$\smallskip \\
			\hspace*{6mm} \textbf{Pre }$\{ True\}$\smallskip \\
			\hspace*{6mm} \textbf{Post }$\{(a<0\rightarrow result=2*b)\vee (
				a\geq 0 \rightarrow result = b-1)\}$\\
			$\}$
		\item
			\textbf{proc f }(in a, b$: \mathbb{R}$,out result$:\mathbb{R})\ \{$\smallskip \\
			\hspace*{6mm} \textbf{Pre }$\{ True\}$\smallskip \\
			\hspace*{6mm} \textbf{Post }$\{result=($if $ a<0$ then $2*b$ else $b-1$ fi $)\}$\\
			$\}$
	\end{enumerate}
\subsection*{Respuesta}
	\begin{enumerate}[label=\alph*)]
		\item Mal, por muchas razones que no tengo ganas de aclarar.
		\item Mal, tendria que ser $a\geq 0$ despues de la conjunción.
		\item Correcta
		\item Correcta
		\item Mmmmmmm.... creo que no, si alguna implicación falla no puedo devolver true.
		\item Correcta
	\end{enumerate}

\paragraph*{Ejercicio 5.}$\bigstar$ Considerar la siguiente especificación, junto con un algoritmo que dado $x$ devuelve $x^2$.\medskip \\
		\textbf{proc unoMasGrande }(in x$: \mathbb{R}$,out result$:\mathbb{R})\ \{$\smallskip \\
		\hspace*{6mm} \textbf{Pre }$\{ True\}$\smallskip \\
		\hspace*{6mm} \textbf{Post }$\{result>x\}$\\
		$\}$
	\begin{enumerate}[label=\alph*)]
		\item ¿Qué devuelve el algoritmo si recibe $x=3$? ¿El resultado hace verdadera la postcondición de \textbf{unoMasGrande}?
		\item ¿Qué sucede para las entradas $x=0.5,x=1,x=0.2$ y $x=-7$?
		\item Teniendo en cuenta lo respondido en los puntos anteriores, escribir una precondición para \textbf{unoMasGrande}, de manera tal que el algoritmo sea una implementación correcta.
	\end{enumerate}
\subsection*{Respuesta}
	\begin{enumerate}[label=\alph*)]
		\item Segun lo que interpreto, el algoritmo esta tratando de cumplir la \textbf{Post}, entonces, al pasarle 3 devuelve un 9 que efectivamente cumple la \textbf{Post} ya que $9>3$. 
		\item 
			\begin{itemize}[label=$\cdot$]
			\item $x=0.5; result =0.25$, no cumple la \textbf{Post}.
			\item $x=1;   result =1$   , no cumple la \textbf{Post}.
			\item $x=0.2; result =0.04$, no cumple la \textbf{Post}.
			\item $x=-7;  result =49$,   cumple la \textbf{Post}.
			\end{itemize}
		\item \textbf{Pre}$\{ abs(x)>1\}$
	\end{enumerate}
	
\paragraph*{Ejercicio 6.}$\bigstar$ Sean $x$ y $r$ variables de tipo $\mathbb{R}$. Considerar los siguientes predicados:
\begin{flalign*}
P1:\{x\leq 0\}& \qquad \qquad \qquad \qquad Q1:\{r\geq x^2\}\\
P2:\{x\leq 10\}& \qquad \qquad \qquad \qquad Q2:\{r\geq 0\}\\
P3:\{x\leq -10\}& \qquad \qquad \qquad \qquad Q3:\{r= x^2\}
\end{flalign*}
	\begin{enumerate}[label=\alph*)]
		\item Indicar la relación de fuerza entre P1, P2 y P3.
		\item Indicar la relación de fuerza entre Q1, Q2 y Q3.
		\item Sea E1 la siguiente especificación. Escribir 2 programas que cumplan con E1.\medskip \\
		\textbf{proc hagoAlgo }(in x$: \mathbb{R}$,out r$:\mathbb{R})\ \{$\smallskip \\
		\hspace*{6mm} \textbf{Pre }$\{ x\leq 0\}$\smallskip \\
		\hspace*{6mm} \textbf{Post }$\{r \geq x^2\}$\\
		$\}$
		\item Sea A un algoritmo que cumple con E1. Decidir si necesariamente cumple las siguientes especificaciones:
		\begin{enumerate}[label=\alph*)]
			\item \textbf{Pre: }$\{x\leq -10\}$, \textbf{Post: }$\{r\geq x^2\}$
			\item \textbf{Pre: }$\{x\leq 10\}$, \textbf{Post: }$\{r\geq x^2\}$
			\item \textbf{Pre: }$\{x\leq 0\}$, \textbf{Post: }$\{r\geq 0\}$
			\item \textbf{Pre: }$\{x\leq 0\}$, \textbf{Post: }$\{r= x^2\}$
			\item \textbf{Pre: }$\{x\leq -10\}$, \textbf{Post: }$\{r\geq 0\}$
			\item \textbf{Pre: }$\{x\leq 10\}$, \textbf{Post: }$\{r\geq 0\}$
			\item \textbf{Pre: }$\{x\leq -10\}$, \textbf{Post: }$\{r= x^2\}$
			\item \textbf{Pre: }$\{x\leq 10\}$, \textbf{Post: }$\{r= x^2\}$
		\end{enumerate}
		\item ¿Qué conclusión pueden sacar? ¿Qué debe cumplirse con respecto a las precondiciones y postcondiciones para que sea seguro reemplazar la especificación?
	\end{enumerate}
\subsection*{Respuesta}
	\begin{enumerate}[label=\alph*)]
		\item \begin{itemize}[label=$\cdot$]
			\item $P1\rightarrow  P2$ es contingencia.
			\item $P1\rightarrow  P3$ es tautologia.
			\item $P2\rightarrow  P1$ es tautologia.
			\item $P2\rightarrow  P3$ es tautologia.
			\item $P3\rightarrow  P1$ es contingencia.
			\item $P3\rightarrow  P2$ es contingencia.
		\end{itemize}	
		\item \begin{itemize}[label=$\cdot$]
			\item $Q1\rightarrow  Q2$ es tautologia.
			\item $Q1\rightarrow  Q3$ es contingencia.
			\item $Q2\rightarrow  Q1$ es contingencia.
			\item $Q2\rightarrow  Q3$ es contingencia.
			\item $Q3\rightarrow  Q1$ es tautologia.
			\item $Q3\rightarrow  Q2$ es tautologia.
		\end{itemize}
		\item \begin{enumerate}[label=\arabic*)]
			\item Programa en lenguaje de especificación:\smallskip \\
				\hspace*{10mm}aux programa1$(x:\mathbb{R})res: \mathbb{R}=x*x+3$
			\item Programa en Perl: \smallskip \\
				\textit{\hspace*{10mm}\# !/usr/bin/perl\\
				\hspace*{10mm}use v5.26;\\
				\hspace*{10mm}my \$ x;\\
				\hspace*{10mm}my \$ res;\\
				\hspace*{10mm}chomp(\$x=$<STDIN>$);\\
				\hspace*{10mm}if(\$$x<=0)\{$\\
				\hspace*{13mm} \$res=$x*x+1$;\\
				\hspace*{10mm}$\}$\\
				\hspace*{10mm}say \$res;
				}
		\end{enumerate}
		\item
			\begin{enumerate}[label=\alph*)]
				\item Cumple.
				\item No cumple.
				\item Cumple.
				\item No Cumple.
				\item Cumple.
				\item No cumple.
				\item No cumple.
				\item No cumple.
			\end{enumerate}
		\item La nueva \textbf{Pre} Tiene que estar incluido en el rango de la
		 		\textbf{Pre} original,y ademas tiene que pasar lo mismo con las \textbf{Post}.
		 
	\end{enumerate}

\paragraph*{Ejercicio 7.}$\bigstar$ Considerar las siguientes dos especificaciones, junto con un algoritmo $a$ que satisface la especificación de \textbf{p2}.\medskip \\
		\textbf{proc p1 }(in x$: \mathbb{R}$,in n$: \mathbb{Z}$,out result$:\mathbb{Z})\ \{$\smallskip \\
		\hspace*{6mm} \textbf{Pre }$\{x\neq 0\}$\smallskip \\
		\hspace*{6mm} \textbf{Post }$\{x^n-1<result\leq x^n\}$\\
		$\}$\medskip \\
		\textbf{proc p2 }(in x$: \mathbb{R}$,in n$: \mathbb{Z}$,out result$:\mathbb{Z})\ \{$\smallskip \\
		\hspace*{6mm} \textbf{Pre }$\{ n\leq 0\rightarrow x\neq 0\}$\smallskip \\
		\hspace*{6mm} \textbf{Post }$\{result=\lfloor x^n\rfloor\}$\\
		$\}$
	\begin{enumerate}[label=\alph*)]
		\item Dados valores de $x$ y $n$ que hacen verdadera la precondición de \textbf{p1}, demostrar que hacen también verdadera la precondición de \textbf{p2}.
		\item Ahora, dados estos valores de $x$ y $n$, supongamos que se ejecuta $a$: llegamos a un valor de $res$ que hace veradadera la postcondición de \textbf{p2}. ¿Será también verdadera la postcondición de \textbf{p1}?
		\item ¿Podemos concluir que $a$ satisface la especificación de \textbf{p1}? 
	\end{enumerate}
\subsection*{Respuesta}
	\begin{enumerate}[label=\alph*)]
		\item
			\begin{align*}
			x\neq 0 \rightarrow (n\leq 0 \rightarrow x\neq 0) &= True\rightarrow 
			(n\leq 0 \rightarrow True)\\
			&= True\rightarrow (True)\\
			&= True
			\end{align*}
		\item Sep.
		\item No, el item anterior valia, porque partiamos del supuesto que las 
			variables cumplian la pre de \textbf{p1}, sin esa restricción entonces ya
			no es verdad que $a$ satisface la especificación de \textbf{p1}.\\
			$n=4$ y $x=0$ cumple \textbf{p2}, pero no cumple \textbf{p1}.
	\end{enumerate}
	
\paragraph*{Ejercicio 8.} Considerar las siguientes especificaciones:\medskip \\
\textbf{proc n-esimo1 }(in l$: seq\langle \mathbb{R}\rangle$,in n$: \mathbb{Z}$,out result$:\mathbb{Z})\ \{$\smallskip \\
		\hspace*{6mm} \textbf{Pre }$\{$\\
		\hspace*{6mm} /*Los elementos están ordenados */\\
		\hspace*{6mm} $(\forall i:\mathbb{Z})(0\leq i < |l|-1
				\rightarrow_L l[i]<l[i+1])$\\
		\hspace*{6mm} $\wedge 0 \leq n < |l|$\\
		\hspace*{6mm} $\}$\\
		\hspace*{6mm} \textbf{Post }$\{result=l[n]\}$\\
		$\}$\medskip \\
		
\textbf{proc n-esimo2 }(in l$: seq\langle \mathbb{R}\rangle$,in n$: \mathbb{Z}$,out result$:\mathbb{Z})\ \{$\smallskip \\
		\hspace*{6mm} \textbf{Pre }$\{$\\
		\hspace*{6mm} /*Los elementos son distintos entre si */\\
		\hspace*{6mm} $(\forall i:\mathbb{Z})(0\leq i < |l|
				\rightarrow_L ((\forall j:\mathbb{Z})(0\leq j < |l|
				\wedge i \neq j)\rightarrow_L l[i]\neq l[j])$\\
		\hspace*{6mm} $\wedge 0 \leq n < |l|$\\
		\hspace*{6mm} $\}$\\
		\hspace*{6mm} \textbf{Post }$\{$\\
		\hspace*{6mm} $resultl\in l$\\
		\hspace*{6mm} $\wedge$\\
		\hspace*{6mm} $n=\sum_{i=0}^{|l|-1}$(if $l[i]<result$ then 1 else 0 fi)\\
		\hspace*{6mm} $\}$\\
		$\}$\medskip \\
	
¿Es cierto que todo algorimo que cumple con \textbf{n-esimo1} cumple también
 con \textbf{n-esimo2}? ¿Y al revéz?\\
 \textbf{Sugerencia:} Razonar de manera análoga a la del ejercicio anterior.
\subsection*{Respuesta}
$l=\langle 1,4,6,7\rangle$, y $n=3$ cumple \textbf{n-esimo1} y \textbf{n-esimo2}.\\
Pero $l=\langle 4,1,6,7\rangle$, y $n=3$ cumple \textbf{n-esimo2} y no \textbf{n-esimo1}.\\
Detalle, encontrar un contraej, me garantiza que no es verdad que si cumple \textbf{n-esimo2}
entonces va a cumplir \textbf{n-esimo1}, sin embargo no ocurre lo mismo para el otro caso.
Lo dejo asi, porque es intuible que es verdad.
	
\paragraph*{Ejercicio 9.}$\bigstar$ Especificar los siguientes problemas:
	\begin{enumerate}[label=\alph*)]
		\item Dado un número entero, decidir si es par.
		\item Dado un entero $n$ y uno $m$, decidir si $n$ es un múltiplo de $m$.
		\item Dadu un número real, devolver su inverso multiplicativo.
		\item Dada una secuencia de caracteres, obtener de ella sólo los que son numéricos (con todas sus apariciones sin importar el orden de aparición).
		\item Dada una secuencia de reales, devolver la secuencia que resulta de duplicar sus valores en las posiciones impares.
		\item Dado un número entero, listar todos sus divisores positivos (sin duplicados).
	\end{enumerate}
\subsection*{Respuesta}
	\begin{enumerate}[label=\alph*)]
		\item
			
			\textbf{proc esPar }(in a$: \mathbb{Z},$out res$: \textbf{Bool})\ \{$\smallskip \\
			\hspace*{6mm} \textbf{Pre }$\{True\}$\smallskip \\
			\hspace*{6mm} \textbf{Post }$\{c=\textrm{a mod }2\}$\\
			$\}$
			
		\item
		
			\textbf{proc esMultiplo }(in n,m$: \mathbb{Z}$,out result$:\textbf{Bool})
			\ \{$\smallskip \\
			\hspace*{6mm} \textbf{Pre }$\{m \neq 0 \vee (m=0\wedge n=0\}$\smallskip \\
			\hspace*{6mm} \textbf{Post }$\{result=(n\textrm{ mod }m=0)\}$\\
			$\}$
		
		\item
			
			\textbf{proc inversoMul}(in x$: \mathbb{R}$,out res$: \mathbb{R})\ \{$\smallskip \\
			\hspace*{6mm} \textbf{Pre }$\{x \neq 0\}$\smallskip \\
			\hspace*{6mm} \textbf{Post }$\{res=x^{-1}\}$\\
			$\}$
			
		\item
		
			\textbf{proc obtenerCaracteres }(in l$: seq\langle \textbf{Char}\rangle$
			out res$: seq\langle \textbf{Char}\rangle)\ \{$\smallskip \\
			\hspace*{6mm} \textbf{Pre }$\{|l|>0\}$\smallskip \\
			\hspace*{6mm} \textbf{Post }$\{$\\
			\hspace*{6mm}$(\forall i:\mathbb{Z})(0\leq i < |l|\wedge_L esDigito(l[i]))$\\
			\hspace*{6mm}$\rightarrow_L$\\
			\hspace*{6mm}$cantAparicionesDeXEnl(l[i],l)=cantAparicionesDeXEnl(l[i],res)$\\
			\hspace*{6mm}$\}$\\
			$\}$\smallskip \\
			
			\textbf{aux cantAparicionesDeXEnl }(c$: \textbf{Char}$
			,l$: seq\langle \textbf{Char}\rangle$)$:\mathbb{Z})$\smallskip \\
			\hspace*{6mm} $=\sum_{i=0}^{|l|-1}$(if $l[i]=c$ then 1 else 0 fi)\\
			
		\item
		
			\textbf{proc DuplicaValoresEnImpares}
			$($in $s: seq\langle \mathbb{R}\rangle,$out $m: seq\langle \mathbb{R}\rangle)\ \{$	\\
					\hspace*{6mm}\textbf{Pre }$\{ |s|>0\}$\\
					\hspace*{6mm}\textbf{Post }$\{$\\
					\hspace*{10mm}$|s|=|m|\wedge_L$\\
					\hspace*{10mm}$(\forall i: \mathbb{Z})(0\leq i <|s|\wedge_L
					\textrm{i mod }2=1)\rightarrow_L(m[i]=s[i]*2)$\\
					\hspace*{10mm}$\wedge(\forall j: \mathbb{Z})(0\leq j <|s|\wedge_L
					\textrm{j mod }2=0)\rightarrow_L(m[i]=s[i])$\\
					\hspace*{6mm}$\}$\\
			$\}$
		\item
			
			\textbf{proc divisoresPositivos }(in a$: \mathbb{Z}$
				,out res$: seq\langle \mathbb{Z}\rangle)\ \{$\smallskip \\
			\hspace*{6mm} \textbf{Pre }$\{a\neq 0\}$\smallskip \\
			\hspace*{6mm} \textbf{Post }$\{(\forall i:\mathbb{Z})
				((1\leq i \leq a\wedge_L \textrm{a mod i}=0)
				\rightarrow_L i\in res)$\\
			\hspace*{6mm}$\wedge_L sinRepetidos(res)\}$\\
			$\}$\smallskip \\
			
			\textbf{pred sinRepetidos}l$: seq\langle \mathbb{Z}\rangle)\ \{$\smallskip \\
			\hspace*{6mm}$(\forall i,j:\mathbb{Z})(0\leq i,j< |l|\rightarrow_L l[i]\neq l[j])$
			$\}$
	\end{enumerate}
	
\paragraph*{Ejercicio 10.} Considerar el problema de decidir, dados $n$ y $m$ enteros, si $n$ es múltiplo de $m$, y la siguiente especificación. \medskip \\
			\textbf{proc esMultiplo }(in n,m$: \mathbb{Z}$,out result$:\textbf{Bool})
			\ \{$\smallskip \\
			\hspace*{6mm} \textbf{Pre }$\{ m \neq 0\}$\smallskip \\
			\hspace*{6mm} \textbf{Post }$\{result=(n\textrm{ mod }m=0)\}$\\
			$\}$
	\begin{enumerate}[label=\alph*)]
		\item Segun la definición matemática de múltiplo, ¿tiene sentido preguntarse si 4
				es múltiplo de 0? ¿Cúal es la respuesta?
		\item ¿Debería ser $n=4, m=0$ una entrada válida para el problema? ¿Lo es en esta
				especificación?
		\item	Corregir la especificación de manera tal que $n=4, m=0$ satisfaga la precondición
				(¡cuidado con las indefiniciones!).
		\item ¿Qué relación de fuerza hay entre la precondición nueva y la original?
	\end{enumerate}
\subsection*{Respuesta}
	\begin{enumerate}[label=\alph*)]
		\item Yo que se, ya no me acuerdo si lo dieron en Algebra I, segun Wolfram Alpha 
				da indefinido.
		\item No, si $m=0$ entonces para cualquier $n$ va a devolver indefinido, salvo para $n=0$,
				ya que el 0 es el unico multiplo de 0.  
		\item \textbf{Pre}$\{m \neq 0 \vee (m=0\wedge n=0) \}$
		\item La original implica la nueva.
	\end{enumerate}

\paragraph*{Ejercicio 11.} Considerar el problema de, dada una secuencia de números reales, devolver la que resulta de duplicar sus valores en las posiciones impares.
	\begin{enumerate}[label=\alph*)]
		\item Para la secuencia $\langle 1,2,3,4\rangle$, ¿es $\langle 0,4,0,8\rangle$
				 un resultado correcto?
		\item Sea la siguiente especificación:\medskip \\
			\textbf{proc duplicarEnImpares }(in l$: seq\langle \mathbb{R}\rangle$
				,out result$:seq\langle \mathbb{R}\rangle )\ \{$\smallskip \\
			\hspace*{6mm} \textbf{Pre }$\{ True\}$\smallskip \\
			\hspace*{6mm} \textbf{Post }$\{|result|=|l|\wedge (\forall i\mathbb{Z})
					(0\leq i < |result|\wedge i\textrm{ mod }2=1)\rightarrow_L 
					result[i]=2*l[i]\}$\\
			$\}$\medskip \\
			Si $l=\langle 1,2,3,4\rangle ,$ ¿$result=\langle 0,4,0,8\rangle$ satisface
			la postcondición?
		\item Si es necesario, corregir la especificación para que describa correctamente
				el resultado esperado.
		\item ¿Qué relación de fuerza hay entre la nueva postcondición y la original?
	\end{enumerate}
\subsection*{Respuesta}
	\begin{enumerate}[label=\alph*)]
		\item Depende de como entiendas el enunciados, asi como esta lo unico que dice es
			que tiene que devolver una lista con los numeros de las posiciones impares de la lista
			de entrada multiplicados por dos, no predica nada mas acerca de que pasa con las 
			posiciones pares, uno puede asumir dos cosa, o que se espera que se mantengan 
			igual a la lista original, o que son irrelevantes y no es necesario respetar
			el contenido de la original.
		\item
			Satisface la postcondición, porque asi como esta, no esta pidiendo nada a las
			posiciones pares, por lo que pueden tomar cualquier valor.
		\item
		
			\textbf{proc duplicarEnImpares }(in l$: seq\langle \mathbb{R}\rangle$
				,out result$:seq\langle \mathbb{R}\rangle )\ \{$\smallskip \\
			\hspace*{6mm} \textbf{Pre }$\{ True\}$\smallskip \\
			\hspace*{6mm} \textbf{Post }$\{$\\
			\hspace*{6mm} $|result|=|l|$\\
			\hspace*{6mm} $\wedge $\\
			\hspace*{6mm} $(\forall i\mathbb{Z})
					(0\leq i < |result|\wedge i\textrm{ mod }2=1)\rightarrow_L 
					result[i]=2*l[i]$\\
			\hspace*{6mm} $\wedge $\\
			\hspace*{6mm} $(\forall j\mathbb{Z})
					(0\leq j < |result|\wedge j\textrm{ mod }2=0)\rightarrow_L 
					result[j]=l[j]$\\
			\hspace*{6mm} $\}$\\
			$\}$\medskip \\
			
		\item La original es más fuerte que la nueva.
	\end{enumerate}
	
\paragraph*{Ejercicio 12.}$\bigstar$ Especificar el problema de dado un entero positivo retornar una secuencia de 0s y 1s que represente es número en base 2 (es decir, en binario).
\subsection*{Respuesta}

			\textbf{proc esBinario }(in numero$:\mathbb{Z}$
				,out result$:seq\langle \mathbb{Z}\rangle )\ \{$\smallskip \\
			\hspace*{6mm} \textbf{Pre }$\{ numero >0\}$\smallskip \\
			\hspace*{6mm} \textbf{Post }$\{$\\
			\hspace*{6mm} $solo0sY1s(result)\wedge$\\
			\hspace*{6mm} $numero=\sum_{i=0}^{|result|-1}result[i]*2^{|result|-1-i}$\\
			\hspace*{6mm} $\}$\\
			$\{$\smallskip \\
			
			\textit{pred solo0sY1s}(l$: seq\langle \mathbb{Z}\rangle)\ \{$\smallskip \\
			\hspace*{6mm}$(\forall i:\mathbb{Z})(0\leq i< |l|\rightarrow_L l[i]=0\vee l[i]=1)$\\
			$\}$
\paragraph*{Ejercicio 13.} Con lo visto en los ejercicios 9 a 12 ¿Encuentra casos de sub y sobreespecificación en las especificaciones del ejercicio 8?
\subsection*{Respuesta}
Dejenme citar a \textbf{Mr. Rodriguez}, \textit{``alta fiacubi, que dice el otro ejercicio?''}.
Despues lo discuto con alguien a este.
	
\paragraph*{Ejercicio 14.} Especificar los siguientes problemas:
	\begin{enumerate}[label=\alph*)]
		\item $\bigstar$ Dado un número entero positivo, obtener la suma de sus factores primos.
		\item Dado un número entero positivo, decidir si es perfecto. Se dice que un número es
				perfecto cuando es igual a la suma de sus divisores (excluyéndose a sí mismo).
		\item Dado un número entero positivo $n$, obtener el menor entero positivo $m>1$ tal que
				$m$ sea coprimo con $n$.
		\item $\bigstar$ Dado un entero positivo, obtener su descomposición en factores primos.
			 	Devolver una secuencia de tuplas ($p,c$), donde $p$ es un factor primo y $e$ es su
			 	exponente, ordenada en forma creciente con respecto a $p$.
		\item Dada una secuencia de números reales, obtener la diferencia máxima entre dos de sus
				elementos.
		\item $\bigstar$ Dada una secuencia de números enteros, devolver aquel que divida a más
				elementos de dicha secuencia. El elemento tiene que pertenecer a la secuencia
				original. Si existe más de un elemento que cumple esta propiedad, devolver alguno 
				de ellos.
	\end{enumerate}
\subsection*{Respuesta}
	\begin{enumerate}[label=\alph*)]
		\item
			
			\textbf{proc sumaFactoresPrimos }(in n$:\mathbb{Z}$
				,out result$: \mathbb{Z} )\ \{$\smallskip \\
			\hspace*{6mm} \textbf{Pre }$\{n>0 \}$\smallskip \\
			\hspace*{6mm} \textbf{Post }$\{$\\
			\hspace*{6mm} $result=\sum_{i=2}^{n}$if $esPrimo(i)$ then $i$ else 0 fi\\
			\hspace*{6mm} $\}$\\
			$\}$\smallskip \\		
			
		\item
			
			\textbf{proc esPerfecto }(in a$:\mathbb{Z}$
				,out result$:\textbf{Bool} )\ \{$\smallskip \\
			\hspace*{6mm} \textbf{Pre }$\{a>0 \}$\smallskip \\
			\hspace*{6mm} \textbf{Post }$\{$\\
			\hspace*{6mm} $a=sumaDivisores(a)$\\
			\hspace*{6mm} $\}$\\
			$\}$\smallskip \\
			
			\textbf{aux sumaDivisores}(n$: \mathbb{Z}):\mathbb{Z}\ \{$\smallskip \\
			\hspace*{6mm}$\sum_{i=1}^{n-1}$ if $n$ mod $i=0$ then $i$ else 0 fi \\
			$\}$	
			
		\item
		
			\textbf{proc menorCoprimo }(in a$:\mathbb{Z}$
				,out m$:\mathbb{Z} )\ \{$\smallskip \\
			\hspace*{6mm} \textbf{Pre }$\{a>0 \}$\smallskip \\
			\hspace*{6mm} \textbf{Post }$\{$\\
			\hspace*{6mm} $m\textrm{ mod }n\neq 0$\\
			\hspace*{6mm} $\wedge$\\
			\hspace*{6mm} $n\textrm{ mod }m\neq 0$\\
			\hspace*{6mm} $\wedge$\\
			\hspace*{6mm} $(\forall i:\mathbb{Z})(0<i<m)\rightarrow n\textrm{ mod }i= 0$\\
			\hspace*{6mm} $\}$\\
			$\}$\smallskip \\
				
			
		\item 
			
			\textbf{proc factores }(in a$:\mathbb{Z}$
				,out result$:seq\langle \mathbb{Z} \times \mathbb{Z}\rangle )\ \{$\smallskip \\
			\hspace*{6mm} \textbf{Pre }$\{ \}$\smallskip \\
			\hspace*{6mm} \textbf{Post }$\{$\\
			\hspace*{6mm} $losValoresDePSonPrimos(result)\wedge$\\
			\hspace*{6mm} $descomposicionCorrecta(,result)\wedge$\\
			\hspace*{6mm} $tuplasOrdenadas(result)$\\
			\hspace*{6mm} $\}$\\
			$\{$\smallskip \\
			
			\textbf{pred losValoresDePSonPrimos}(l$:seq\langle \mathbb{Z} \times \mathbb{Z}
			\rangle)\ \{$\smallskip \\
			\hspace*{6mm}$(\forall elem:\mathbb{Z}\times \mathbb{Z})
				(elem\in result\rightarrow esPrimo((elem)_0))$\\
			$\}$\smallskip \\
			
			\textbf{pred descomposicionCorrecta}(a $:\mathbb{Z}$,
			l$:seq\langle \mathbb{Z} \times \mathbb{Z}\rangle)\ 
			\{$\smallskip \\
			\hspace*{6mm}$a=\prod_{i=0}^{|result|-1}(result[i])_{0}^{(result[i])_1}$\\
			$\}$\smallskip \\	
			
			\textbf{pred tuplasOrdenadas}(l$:seq\langle \mathbb{Z} \times \mathbb{Z}
			\rangle)\ \{$\smallskip \\
			\hspace*{6mm}$(\forall i:\mathbb{Z})(0\leq i <|result|-1\rightarrow_L 
				(result[i])_0<(result[i+1])_0)$\\
			$\}$\smallskip \\	
			
		\item
			
			\textbf{proc difMax}(in l$:seq\langle \mathbb{R}\rangle$,out res$:\mathbb{R} )\ 
			\{$\smallskip \\
			\hspace*{6mm} \textbf{Pre }$\{ |l|>1\}$\smallskip \\
			\hspace*{6mm} \textbf{Post }$\{$\\
			\hspace*{6mm} $(\exists e1,e2:\mathbb{R})(e1,e2\in l \wedge res=abs(e1-e2))$\\
			\hspace*{6mm} $\wedge$\\
			\hspace*{6mm} $(\forall i,j:\mathbb{R})(i,j\in l \wedge \neg(i=e1 \wedge j=e2)
			\wedge \neg(i=e2 \wedge j=e1))\rightarrow abs(j-i)<abs(e1-e2)$\\
			\hspace*{6mm} $\}$\\
			$\{$\smallskip \\
			
			\textbf{pred pred1}(l$: seq\langle \mathbb{Z}\rangle)\ \{$\smallskip \\
			\hspace*{6mm}$body$\\
			$\}$	
			
		\item
			
			\textbf{proc DivideMas}(in l$:seq\langle \mathbb{Z}\rangle$,out res$:\mathbb{Z} )\ 
			\{$\smallskip \\
			\hspace*{6mm} \textbf{Pre }$\{ |l|>0\}$\smallskip \\
			\hspace*{6mm} \textbf{Post }$\{$\\
			\hspace*{6mm} $res\in l$\\
			\hspace*{6mm} $\wedge$\\
			\hspace*{6mm} $(\forall elem:\mathbb{Z})(elem \in l\wedge elem\neq res)\rightarrow
			cantDivEnLporN(res,l)>cantDivEnLporN(elem,l)$\\
			\hspace*{6mm} $\}$\\
			$\{$\smallskip \\
			
			\textbf{aux cantDivEnLporN}($n:\mathbb{Z},l: seq\langle \mathbb{Z}\rangle):\mathbb{Z}
			\ \{$\smallskip \\
			\hspace*{6mm}$\sum_{i=0}^{|l|-1}$ if $l[i]$ mod $n=0$ then 1 else 0 fi\\
			$\}$	
	\end{enumerate}
	
\paragraph*{Ejercicio 15.} Especificar los siguientes problemas sobre secuencias:
	\begin{enumerate}[label=\alph*)]
		\item \textbf{proc nEsimaAparicion}(in $l:seq\langle \mathbb{R}\rangle$,
			in $e:\mathbb{R}$,in $n:\mathbb{Z}$,out $result:\mathbb{Z}$), que devuelve
			el índice de la n-ésima aparición de $e$ en $l$.
		\item Dadas dos secuencias $s$ y $t$, decidir si $s$ es una subcadena de $t$.
		\item $\bigstar$ Dadas dos secuencias $s$ y $t$, decidir si $s$ está $incluida$ 
		en $t$, es decir,si todos los elementos de $s$ aparecen en $t$ en igual o mayor cantidad.
		\item \textbf{proc mezclarOrdenado}(in $s,t:seq\langle \mathbb{Z}\rangle$,
		out $result:seq\langle \mathbb{Z}\rangle$) que recibe dos secuencias ordenadas y devuelve
			el resultado de intercalar sus elementos de manera ordenada.
		\item Dadas dos secuencias $s$ y $t$ especificar el procedimiento 
				\textit{intersecciónSinRepetidos} que retorna una secuencia que contiene
				únicamente los elementos que aparecen en ambas secuencias.
		\item $\bigstar$ Dadas dos secuencias $s$ y $t$, devolver su \textit{intersección}, es
				decir, una secuencia con todos los elemntos que aparecen en ambas. Si un mismo
				elemento tiene repetidos, la secuencia retornada debe contener la cantidad
				mínima de apariciones en de $s$ y de $t$.
	\end{enumerate}
\subsection*{Respuesta}
	\begin{enumerate}[label=\alph*)]
		\item
			
			\textbf{proc nEsimaAparicion }(in l$:seq\langle \mathbb{R}\rangle$,in $e:\mathbb{R}$
			,in $n:\mathbb{Z}$,out $result:\mathbb{Z})\ \{$\smallskip \\
			\hspace*{6mm} \textbf{Pre }$\{ cantApariciones(e,l)\geq n\}$\smallskip \\
			\hspace*{6mm} \textbf{Post }$\{$\\
			\hspace*{6mm} $n=cantApariciones(subseq(l,0,l[res]),e)$\\
			\hspace*{6mm} $\}$\\
			$\}$\smallskip \\
			
			\textbf{aux cantApariciones}($e:\mathbb{R},l: seq\langle \mathbb{R}\rangle):\mathbb{Z}
			\ \{$\smallskip \\
			\hspace*{6mm}$\sum_{i=0}^{|l|-1}$ if $e=l[i]$ then 1 else 0 fi\\
			$\}$			
			
		\item
			
			\textbf{proc esSubcadena }(in $l:seq\langle \mathbb{Z}\rangle$
				,in $m:seq\langle \mathbb{Z}\rangle$,out res$\textbf{Bool} )\ \{$\smallskip \\
			\hspace*{6mm} \textbf{Pre }$\{ \}$\smallskip \\
			\hspace*{6mm} \textbf{Post }$\{$\\
			\hspace*{6mm} $prd1(a)$\\
			\hspace*{6mm} $\wedge$\\
			\hspace*{6mm} $prd2(a)$\\
			\hspace*{6mm} $\}$\\
			$\{$\smallskip \\
			
			\textbf{pred pred1}(l$: seq\langle \mathbb{Z}\rangle)\ \{$\smallskip \\
			\hspace*{6mm}$body$\\
			$\}$	
			
		\item
		
			\textbf{proc estaIncluilda }(in $l:seq\langle \mathbb{Z}\rangle$
				,in $m:seq\langle \mathbb{Z}\rangle$,out res$:\textbf{Bool} )\ \{$\smallskip \\
			\hspace*{6mm} \textbf{Pre }$\{|l| \leq |m|\}$\smallskip \\
			\hspace*{6mm} \textbf{Post }$\{$\\
			\hspace*{6mm} $res=(\forall elem:\mathbb{Z})(elem\in l)\rightarrow 
			\#apariciones(elem,l)\leq \#apariciones(elem,m)$\\
			\hspace*{6mm} $\}$\\
			$\{$\smallskip \\
			
			\textbf{aux \#apariciones}($e:\mathbb{Z},l: seq\langle \mathbb{Z}\rangle):\mathbb{Z}
			\ \{$\smallskip \\
			\hspace*{6mm}$\sum_{i=0}^{|l|-1}$ if $e=l[i]$ then 1 else 0 fi\\
			$\}$
			
		\item
			
			\textbf{proc mezclarOrdenado }(in $l:seq\langle \mathbb{Z}\rangle$
				,out $res:seq\langle \mathbb{Z}\rangle )\ \{$\smallskip \\
			\hspace*{6mm} \textbf{Pre }$\{ estaOrdenada(l)\wedge
			 estaOrdenada(m)\}$\smallskip \\
			\hspace*{6mm} \textbf{Post }$\{$\\
			\hspace*{6mm} $incluidoEnAlguno(l,m,res)$\\
			\hspace*{6mm} $\wedge$\\
			\hspace*{6mm} $mismaCantRep(l,m,res)$\\
			\hspace*{6mm} $\wedge$\\
			\hspace*{6mm} $estaOrdenada(res)$\\
			\hspace*{6mm} $\}$\\
			$\{$\smallskip \\
			
			\textbf{pred incluidoEnAlguno}($l,m,res:seq\langle T\rangle)
			\ \{$\smallskip \\
			\hspace*{6mm}$ (\forall elem:T)(elem\in res) \leftrightarrow (elem\in l \vee 
			elem\in m)$\\
			$\}$\smallskip \\	
			
			\textbf{pred mismaCantRep}($l,m,res:seq\langle T\rangle)
			\ \{$\smallskip \\
			\hspace*{6mm}$(\forall j:\mathbb{Z})(0\leq j<|res|)\rightarrow_L
			cantRep(res,res[j])=sumRep(l,m,res[j])$\\
			$\}$\smallskip \\	
			
			\textbf{aux sumRep}($l,m:seq\langle T\rangle$
				,$n:T ):\mathbb{Z}\ \{$\smallskip \\
			\hspace*{6mm} $cantRep(l,n)+cantRep(m,n)$\\
			$\}$\smallskip \\			
			
			\textbf{aux cantRep}(l$: seq\langle T\rangle$, $n:T):\mathbb{Z}
			\ \{$\smallskip \\
			\hspace*{6mm}$\sum_{i=0}^{|l|-1}$ if $l[i]=n$ then 1 else 0 fi\\
			$\}$\smallskip \\
			
			\textbf{pred estaOrdenada}$(s: seq\langle \mathbb{Z} \rangle)\{$\\
			\hspace*{6mm}$(\forall i:\mathbb{Z})
				(0\leq i<\vert s\vert -1\rightarrow_L s[i]\leq s[i+1])$\\
			$\}$\smallskip \\
			
		\item
			
			\textbf{proc interseccionSinRepetidos }(in $l:seq\langle T\rangle$
				,in $m:seq\langle T\rangle$
				,out $res:seq\langle T\rangle )\ \{$\smallskip \\
			\hspace*{6mm} \textbf{Pre }$\{ True\}$\smallskip \\
			\hspace*{6mm} \textbf{Post }$\{$\\
			\hspace*{6mm} $incluidoEnAmbos(l,m,res)$\\
			\hspace*{6mm} $\wedge_L$\\
			\hspace*{6mm} $sinRep(res)$\\
			\hspace*{6mm} $\}$\\
			$\{$\smallskip \\
			
			\textbf{pred incluidoEnAmbos}($l,m,res:seq\langle T\rangle)
			\ \{$\smallskip \\
			\hspace*{6mm}$ (\forall elem:T)(elem\in res) \leftrightarrow (elem\in l \wedge 
			elem\in m)$\\
			$\}$\smallskip \\	
			
			\textbf{pred sinRep}($res:seq\langle T\rangle)
			\ \{$\smallskip \\
			\hspace*{6mm}$(\forall j:\mathbb{Z})(0\leq j<|res|)\rightarrow_L
			cantRep(res,res[j])=1$\\
			$\}$\smallskip \\	
						
			
			\textbf{aux cantRep}(l$: seq\langle T\rangle$, $n:T):\mathbb{Z}
			\ \{$\smallskip \\
			\hspace*{6mm}$\sum_{i=0}^{|l|-1}$ if $l[i]=n$ then 1 else 0 fi\\
			$\}$\smallskip \\	
			
		\item
			
			\textbf{proc interseccion }(in $l:seq\langle T\rangle$
				,in $m:seq\langle T\rangle$
				,out $res:seq\langle T\rangle )\ \{$\smallskip \\
			\hspace*{6mm} \textbf{Pre }$\{ True\}$\smallskip \\
			\hspace*{6mm} \textbf{Post }$\{$\\
			\hspace*{6mm} $incluidoEnAmbos(l,m,res)$\\
			\hspace*{6mm} $\wedge_L$\\
			\hspace*{6mm} $mismaCantRep(l,m,res)$\\
			\hspace*{6mm} $\}$\\
			$\{$\smallskip \\
			
			\textbf{pred incluidoEnAmbos}($l,m,res:seq\langle T\rangle)
			\ \{$\smallskip \\
			\hspace*{6mm}$ (\forall elem:T)(elem\in res) \leftrightarrow (elem\in l \wedge 
			elem\in m)$\\
			$\}$\smallskip \\	
			
			\textbf{pred mismaCantRep}($l,m,res:seq\langle T\rangle)
			\ \{$\smallskip \\
			\hspace*{6mm}$(\forall j:\mathbb{Z})(0\leq j<|res|)\rightarrow_L
			cantRep(res,res[j])=minRep(l,m,res[j])$\\
			$\}$\smallskip \\	
			
			\textbf{aux minRep}($l,m:seq\langle T\rangle$
				,$n:T ):\mathbb{Z}\ \{$\smallskip \\
			\hspace*{6mm}if $cantRep(l,n)<cantRep(m,n)$
				 then $cantRep(l,n)$ else $cantRep(m,n)$ fi\\
			$\}$\smallskip \\			
			
			\textbf{aux cantRep}(l$: seq\langle T\rangle$, $n:T):\mathbb{Z}
			\ \{$\smallskip \\
			\hspace*{6mm}$\sum_{i=0}^{|l|-1}$ if $l[i]=n$ then 1 else 0 fi\\
			$\}$\smallskip \\	
			
	\end{enumerate}
	
\paragraph*{Ejercicio 16.} Especificar los siguientes problemas:
	\begin{enumerate}[label=\alph*)]
		\item \textbf{proc cantApariciones}(in $l:\textbf{String},$out $result :
			 seq(\textbf{Char}\times \mathbb{Z})$ que devuelve la secuencia con todos los elementos de $l$, sin duplicados con su cantidad de apariciones (en un orden cualquiera).
			  Ejemplos: \\
			 \begin{itemize}
			 	\item \textit{cantApariciones}
			 	$(\langle 'a'\rangle)=\langle \langle 'a',1\rangle \rangle$
			 	\item \textit{cantApariciones}
			 	$(\langle 'a','b','c'\rangle)=\langle \langle 'a',1\rangle ,\langle 'c',1\rangle
			 		,\langle 'b',1\rangle \rangle$
			 	\item \textit{cantApariciones}
			 	$(\langle 'a','b','c','b','d','b'\rangle)=\langle \langle 'a',1\rangle, 
			 	\langle 'b',3\rangle ,\langle 'd',1\rangle ,\langle 'c',1\rangle \rangle$
			 	\item \textit{cantApariciones}
			 	$(\langle \rangle)=\langle \rangle$
			 \end{itemize}
		\item Dada una secuencia, devolver una secuencia de secuencias que contenga todos sus
				prefijos, en orden creciente de longitud.
		\item $\bigstar$ Dada una secuencia de secuencias de enteros l, devolver una secuencia de
				l que contenga el máximo valor. Por ejemplo, si $l=\langle 
				\langle 2,3,5\rangle , \langle 8,1\rangle , \langle 2,8,4,3\rangle\rangle$, 
				devolver $\langle 8,1\rangle$ o $\langle 2,8,4,3\rangle$.
		\item \textbf{proc interseccionMultiple}(in $ls:seq\langle seq\langle \mathbb{Z}\rangle
		 \rangle ,$out $l: seq(\langle \mathbb{R}\rangle)$ que devuelve en $l$ el resultado de 
		 		la intersección de todas las secuencias de $ls$.
		\item $\bigstar$ Dada una secuencia $l$ con todos sus elementos distintos, devolver la
				secuencia de $partes$, es decir, la secuencia de todas las secuencias incluidas
				en $l$, cada una con sus elementos en el mismo orden en que aparecen en $l$.
	\end{enumerate}
\subsection*{Respuesta}
	\begin{enumerate}[label=\alph*)]
		\item
			
			\textbf{proc cantApariciones}(in $l:\textbf{String},$out $result :
			 seq(\textbf{Char}\times \mathbb{Z})\ \{$\smallskip \\
			\hspace*{6mm} \textbf{Pre }$\{ \}$\smallskip \\
			\hspace*{6mm} \textbf{Post }$\{$\\
			\hspace*{6mm} $prd1(a)$\\
			\hspace*{6mm} $\wedge$\\
			\hspace*{6mm} $prd2(a)$\\
			\hspace*{6mm} $\}$\\
			$\{$\smallskip \\
			
			\textbf{pred pred1}(l$: seq\langle \mathbb{Z}\rangle)\ \{$\smallskip \\
			\hspace*{6mm}$body$\\
			$\}$			
			
		\item
			
			\textbf{proc prefijos }(in l$:seq\langle \mathbb{Z}\rangle$
				,out result$:seq\langle \mathbb{Z}\rangle )\ \{$\smallskip \\
			\hspace*{6mm} \textbf{Pre }$\{ \}$\smallskip \\
			\hspace*{6mm} \textbf{Post }$\{$\\
			\hspace*{6mm} $prd1(a)$\\
			\hspace*{6mm} $\wedge$\\
			\hspace*{6mm} $prd2(a)$\\
			\hspace*{6mm} $\}$\\
			$\{$\smallskip \\
			
			\textbf{pred pred1}(l$: seq\langle \mathbb{Z}\rangle)\ \{$\smallskip \\
			\hspace*{6mm}$body$\\
			$\}$	
			
		\item
		
			\textbf{proc MaxValor }(out result$:seq\langle seq\langle \mathbb{Z}\rangle \rangle$
				,out result$:seq\langle \mathbb{Z}\rangle )\ \{$\smallskip \\
			\hspace*{6mm} \textbf{Pre }$\{ \}$\smallskip \\
			\hspace*{6mm} \textbf{Post }$\{$\\
			\hspace*{6mm} $prd1(a)$\\
			\hspace*{6mm} $\wedge$\\
			\hspace*{6mm} $prd2(a)$\\
			\hspace*{6mm} $\}$\\
			$\{$\smallskip \\
			
			\textbf{pred pred1}(l$: seq\langle \mathbb{Z}\rangle)\ \{$\smallskip \\
			\hspace*{6mm}$body$\\
			$\}$	
			
		\item
			
			\textbf{proc interseccionMultiple}(in $ls:seq\langle seq\langle \mathbb{Z}\rangle
			 \rangle ,$out $l: seq\langle \mathbb{R}\rangle)\ \{$\smallskip \\
			\hspace*{6mm} \textbf{Pre }$\{ \}$\smallskip \\
			\hspace*{6mm} \textbf{Post }$\{$\\
			\hspace*{6mm} $prd1(a)$\\
			\hspace*{6mm} $\wedge$\\
			\hspace*{6mm} $prd2(a)$\\
			\hspace*{6mm} $\}$\\
			$\{$\smallskip \\
			
			\textbf{pred pred1}(l$: seq\langle \mathbb{Z}\rangle)\ \{$\smallskip \\
			\hspace*{6mm}$body$\\
			$\}$		
			
		\item
		
			\textbf{proc partesDeDistintos}(in $l: seq \langle \mathbb{Z}\rangle$
				,out $res:seq\langle seq\langle \mathbb{Z}\rangle \rangle \ \{$\smallskip \\
			\hspace*{6mm} \textbf{Pre }$\{ \}$\smallskip \\
			\hspace*{6mm} \textbf{Post }$\{$\\
			\hspace*{6mm} $prd1(a)$\\
			\hspace*{6mm} $\wedge$\\
			\hspace*{6mm} $prd2(a)$\\
			\hspace*{6mm} $\}$\\
			$\{$\smallskip \\
			
			\textbf{pred pred1}(l$: seq\langle \mathbb{Z}\rangle)\ \{$\smallskip \\
			\hspace*{6mm}$body$\\
			$\}$	
	\end{enumerate}
	
\section*{Especificación de problemas usando \textbf{inout}}

\paragraph*{Ejercicio 17.} $\bigstar$ Dados dos enteros $a$ y $b$, se necesita calcular su suma y retornarla en un entero $c$. ¿Cúales de las siguientes especificaciones son correctas para este problema? Para las que no lo son, indicar por qué. 
	\begin{enumerate}[label=\alph*)]
		\item
			\textbf{proc sumar }(inout a,b,c$: \mathbb{Z})\ \{$\smallskip \\
			\hspace*{6mm} \textbf{Pre }$\{True\}$\smallskip \\
			\hspace*{6mm} \textbf{Post }$\{a+b=c\}$\\
			$\}$
		\item
			\textbf{proc sumar }(in a,b$: \mathbb{Z}$,in c$: \mathbb{Z})\ \{$\smallskip \\
			\hspace*{6mm} \textbf{Pre }$\{ True\}$\smallskip \\
			\hspace*{6mm} \textbf{Post }$\{c=a+b\}$\\
			$\}$
		\item
			\textbf{proc sumar }(in a,b$: \mathbb{Z}$,out c$: \mathbb{Z})\ \{$\smallskip \\
			\hspace*{6mm} \textbf{Pre }$\{ True\}$\smallskip \\
			\hspace*{6mm} \textbf{Post }$\{c=a+b\}$\\
			$\}$
		\item
			\textbf{proc sumar }(inout a,b$: \mathbb{Z}$,out c$: \mathbb{Z})\ \{$\smallskip \\
			\hspace*{6mm} \textbf{Pre }$\{ a=A_0\wedge b=B_0\}$\smallskip \\
			\hspace*{6mm} \textbf{Post }$\{ a=A_0\wedge b=B_0\wedge c=a+b\}$\\
			$\}$
	\end{enumerate}
\subsection*{Respuesta}
	\begin{enumerate}[label=\alph*)]
		\item Incorrecta, porque puedo alterar los valores de $a$ y $b$ para 
			que cumplan la ecuación de la \textbf{Post}.
		\item Incorrecta, porque no puedo devolver $c$.
		\item Correcta.
		\item Correcta.
	\end{enumerate}
	
\paragraph*{Ejercicio 18.}$\bigstar$ Dada una secuencia $l$, se desea sacar su primer elemento y devolverlo. Decidir cúales de estas especificaciones son correctas. Para las que no lo son, indicar por qué y justificar con ejemplos.
	\begin{enumerate}[label=\alph*)]
		\item
			\textbf{proc tomarPrimero }(inout l$: seq\langle \mathbb{R}\rangle $
			,out result$:\mathbb{R})\ \{$\smallskip \\
			\hspace*{6mm} \textbf{Pre }$\{ |l|> 0\}$\smallskip \\
			\hspace*{6mm} \textbf{Post }$\{result=\textrm{head}(l)\}$\\
			$\}$
		\item
			\textbf{proc tomarPrimero }(inout l$: seq\langle \mathbb{R}\rangle $
			,out result$:\mathbb{R})\ \{$\smallskip \\
			\hspace*{6mm} \textbf{Pre }$\{ |l|> 0\wedge l=L_0\}$\smallskip \\
			\hspace*{6mm} \textbf{Post }$\{result=\textrm{head}(L_0)\}$\\
			$\}$
		\item
			\textbf{proc tomarPrimero }(inout l$: seq\langle \mathbb{R}\rangle $
			,out result$:\mathbb{R})\ \{$\smallskip \\
			\hspace*{6mm} \textbf{Pre }$\{ |l|> 0\}$\smallskip \\
			\hspace*{6mm} \textbf{Post }$\{result=\textrm{head}(L_0)\wedge
					|l|=|L_0|-1\}$\\
			$\}$
		\item
			\textbf{proc tomarPrimero }(inout l$: seq\langle \mathbb{R}\rangle $
			,out result$:\mathbb{R})\ \{$\smallskip \\
			\hspace*{6mm} \textbf{Pre }$\{ |l|> 0\wedge l=L_0\}$\smallskip \\
			\hspace*{6mm} \textbf{Post }$\{result=\textrm{head}(L_0)\wedge
					l=\textrm{tail}(L_0)\}$\\
			$\}$
		\item 
			\textbf{proc tomarPrimero }(inout l$: seq\langle \mathbb{R}\rangle $
			,out result$:\mathbb{R})\ \{$\smallskip \\
			\hspace*{6mm} \textbf{Pre }$\{ |l|> 0\wedge l=L_0\}$\smallskip \\
			\hspace*{6mm} \textbf{Post }$\{$\\
			\hspace*{6mm} $result=\textrm{head}(L_0)$\\
			\hspace*{6mm} $\wedge |l|=|L_0|-1$\\
			\hspace*{6mm} $\wedge_L (\forall i:\mathbb{Z})(0\leq i < |l|
						\rightarrow_L l[i]=L_0[i+1]))$\\
			\hspace*{6mm}$\}$\\
			$\}$
	\end{enumerate}
\subsection*{Respuesta}
	\begin{enumerate}[label=\alph*)]
		\item No dice nada acerca de que pasa con $l$ en la \textbf{Post}
		\item Sigue sin decir nada acerca de que pasa con $l$ en la \textbf{Post}
		\item Que $l$ tenga un elemento menos que $L_0$ no implica que sea el primero.
		\item Correcta.
		\item Correcta.
	\end{enumerate}

\paragraph*{Ejercicio 19.} Considerar la siguiente especificación:\medskip \\

			\textbf{proc intercambiar }(inout l$: seq\langle
				 \mathbb{R}\rangle$,in $i,j:\mathbb{Z})
			\ \{$\smallskip \\
			\hspace*{6mm} \textbf{Pre }$\{0\leq i<|l|
					\wedge 0\leq j<|l|\wedge l=L_0\}$\\
			\hspace*{6mm} \textbf{Post }$\{$\\
			\hspace*{6mm} /*Las secuencias tienen la misma longitud*/\\
			\hspace*{6mm} $|l|=|L_0|$\\
			\hspace*{6mm} $\wedge$\\
			\hspace*{6mm} /*Intercambia i*/\\
			\hspace*{6mm} $l[i]=L_0[j]$\\
			\hspace*{6mm} $\wedge$\\
			\hspace*{6mm} /*Intercambia j*/\\
			\hspace*{6mm} $l[j]=L_0[i]$\\
			\hspace*{6mm} $\}$\\
			$\}$
			
	\begin{enumerate}[label=\alph*)]
		\item ¿Esta especificación es válida? Si lo es ¿qué problema describe?
		\item Mostrar con un ejemplo que la postcondicion está sub-especificada (es decir, que 
				hay valores que la hacen verdadera aunque no son deseables como solución).
		\item Corregir la especificación agregando a la postcondición una o más cláusulas
			 \textbf{Post: }.
	\end{enumerate}
\subsection*{Respuesta}
	\begin{enumerate}[label=\alph*)]
		\item Mmm, mira con la cara que te mira Conan, para mi que hay tramuyo aca.
		\item
		\item
	\end{enumerate}
	
\paragraph*{Ejercicio 20.} Explicar coloquialmente la siguiente especificación:\medskip \\
			\textbf{proc copiarPrimero }(inout l$: seq\langle
				 \mathbb{R}\rangle$,inout i$:\mathbb{Z})
			\ \{$\smallskip \\
			\hspace*{6mm} \textbf{Pre }$\{$\\
			\hspace*{6mm} /*Valores iniciales*/\\
			\hspace*{6mm} $l=L_0\wedge i=I_0$\\
			\hspace*{6mm} $\wedge$\\
			\hspace*{6mm} /*Secuencia no vacia*/\\
			\hspace*{6mm} $|l|>0$\\
			\hspace*{6mm} $\wedge$\\
			\hspace*{6mm} /*Indice en rango*/\\
			\hspace*{6mm} $0\leq i <|l|$\\
			\hspace*{6mm} $\}$\\
			\hspace*{6mm} \textbf{Post }$\{$\\
			\hspace*{6mm} $l[I_0]=L_0[0]$\\
			\hspace*{6mm} $\wedge$\\
			\hspace*{6mm} $i=L_0[I_0]$\\
			\hspace*{6mm} $\wedge$\\
			\hspace*{6mm} $((\forall j:\mathbb{Z})(0\leq j<|l|\wedge j\neq I_0)
							\rightarrow_L l[j]=L_0[I_0])$\\
			\hspace*{6mm} $\}$\\
			$\}$
	
\subsection*{Respuesta}
	
	
\paragraph*{Ejercicio 21.} Dada una secuencia de enteros, se requiere multiplicar por 2 aquéllos valores que se encuentran en posiciones pares. Indicar por qué son incorrectas las siguientes especificaciones, y proponer una alternativa correcta
	\begin{enumerate}[label=\alph*)]
		\item
			
			\textbf{proc duplicarPares }(inout l$: seq\langle \mathbb{Z}\rangle)\ \{$\smallskip \\
			\hspace*{6mm} \textbf{Pre }$\{ l=L_0\}$\smallskip \\
			\hspace*{6mm} \textbf{Post }$\{$\\
			\hspace*{6mm} $|l|=|L_0|$\\
			\hspace*{6mm} $\wedge$\\
			\hspace*{6mm} $(\forall i:\mathbb{Z})(0\leq i <|l|\wedge i \textrm{ mod }2=0)
							\rightarrow_L l[i]=2*L_0[i]$\\			
			\hspace*{6mm}$\}$\\
			$\}$
		\item
			\textbf{proc duplicarPares }(inout l$: seq\langle \mathbb{Z}\rangle)\ \{$\smallskip \\
			\hspace*{6mm} \textbf{Pre }$\{ l=L_0\}$\smallskip \\
			\hspace*{6mm} \textbf{Post }$\{(\forall i:\mathbb{Z})
					(0\leq i <|l|\wedge i \textrm{ mod }2\neq 0)
							\rightarrow_L l[i]=L_0[i]$\\
			\hspace*{6mm} $\wedge$\\
			\hspace*{6mm} $(\forall i:\mathbb{Z})(0\leq i <|l|\wedge i \textrm{ mod }2=0)
							\rightarrow_L l[i]=2*L_0[i]$\\				
			\hspace*{6mm}$\}$\\
			$\}$
		\item
			\textbf{proc duplicarPares }(inout l$: seq\langle \mathbb{Z}\rangle
				$,out result$:seq\langle \mathbb{Z}\rangle)\ \{$\smallskip \\
			\hspace*{6mm} \textbf{Pre }$\{ True\}$\smallskip \\
			\hspace*{6mm} \textbf{Post }$\{|l|=|result|$\\
			\hspace*{6mm} $\wedge$\\
			\hspace*{6mm} $(\forall i:\mathbb{Z})(0\leq i <|l|\wedge i \textrm{ mod }2\neq 0)
							\rightarrow_L result[i]=l[i]$\\	
			\hspace*{6mm} $\wedge$\\
			\hspace*{6mm} $(\forall i:\mathbb{Z})(0\leq i <|l|\wedge i \textrm{ mod }2=0)
							\rightarrow_L result[i]=2*l[i]$\\			
			\hspace*{6mm}$\}$\\
			$\}$
	\end{enumerate}
\subsection*{Respuesta}
	\begin{enumerate}[label=\alph*)]
		\item
		\item
		\item
	\end{enumerate}
	
\paragraph*{Ejercicio 22.} Especificar los siguientes problemas de modificación de secuencias:
	\begin{enumerate}[label=\alph*)]
		\item $\bigstar$ \textbf{proc primosHermanos}(inout $l:seq\langle \mathbb{Z}\rangle$), que dada una secuencia de enteros mayores a dos, reemplaza dichos valores por el número primo menor más cercano. Por ejemplo, si $l=\langle 6,5,9,14 \rangle$, luego de aplicar \textbf{primosHermanos}($l$), $l=\langle 5,5,7,13 \rangle$
		\item $\bigstar$ \textbf{proc reemplazar}(inout $l:$\textbf{String}, in $a,b:$\textbf{Char}), que reemplaza todas las apariciones de $a$ en $l$ por $b$.
		\item \textbf{proc recortar}(inout $l:seq\langle \mathbb{Z}\rangle$, in $a:\mathbb{Z}$), que saca de $l$ todas las apariciones de $a$ consecutivas que aparezcan al principio. Por ejemplo \textbf{recortar}($\langle 2,2,3,2,4\rangle ,2)=\langle 3,2,4\rangle$, mientras que \textbf{recortar}($\langle 2,2,3,2,4\rangle ,3)=\langle 2,2,3,2,4\rangle$.
		\item \textbf{proc intercambiarParesConImpares}(inout $l:$\textbf{String}), que toma una secuencia de longitud par y la modifica de modo tal que todas las posciciones de la forma $2k$ quedan intercambiadas con las posiciones $2k+1$. Por ejemplo, \textbf{intercambiarParesConImpares}(\textit{``adinle"}) modifica de la siguiente manera: \textit{``daniel"}.
		\item $\bigstar$ \textbf{proc limpiarDuplicados}(inout $l: seq\langle \mathbb{\textbf{Char}}\rangle$, out $dup:seq\langle \mathbb{\textbf{Char}}\rangle$), que elimina los elementos duplicados de $l$ dejando sólo su primera aparición (en el orden original). Devuelve además, $dup$ una secuencia con todas las apariciones eliminadas (en cualquier orden).
	\end{enumerate}
\subsection*{Respuesta}
	\begin{enumerate}[label=\alph*)]
		\item
			
			\textbf{proc primosHermanos}(inout $l:seq\langle \mathbb{Z}\rangle$) $\{$\smallskip \\
			\hspace*{6mm} \textbf{Pre }$\{ \}$\smallskip \\
			\hspace*{6mm} \textbf{Post }$\{$\\
			\hspace*{6mm} $prd1(a)$\\
			\hspace*{6mm} $\wedge$\\
			\hspace*{6mm} $prd2(a)$\\
			\hspace*{6mm} $\}$\\
			$\{$\smallskip \\
			
			\textbf{pred pred1}(l$: seq\langle \mathbb{Z}\rangle)\ \{$\smallskip \\
			\hspace*{6mm}$body$\\
			$\}$			
			
		\item
			
			\textbf{proc reemplazar}(inout $l:$\textbf{String}, in $a,b:$\textbf{Char}) 
			$\{$\smallskip \\
			\hspace*{6mm} \textbf{Pre }$\{ \}$\smallskip \\
			\hspace*{6mm} \textbf{Post }$\{$\\
			\hspace*{6mm} $prd1(a)$\\
			\hspace*{6mm} $\wedge$\\
			\hspace*{6mm} $prd2(a)$\\
			\hspace*{6mm} $\}$\\
			$\{$\smallskip \\
			
			\textbf{pred pred1}(l$: seq\langle \mathbb{Z}\rangle)\ \{$\smallskip \\
			\hspace*{6mm}$body$\\
			$\}$	
			
		\item
		
			\textbf{proc recortar}(inout $l:seq\langle \mathbb{Z}\rangle$, in $a:\mathbb{Z}$)
			$\{$\smallskip \\
			\hspace*{6mm} \textbf{Pre }$\{ \}$\smallskip \\
			\hspace*{6mm} \textbf{Post }$\{$\\
			\hspace*{6mm} $prd1(a)$\\
			\hspace*{6mm} $\wedge$\\
			\hspace*{6mm} $prd2(a)$\\
			\hspace*{6mm} $\}$\\
			$\{$\smallskip \\
			
			\textbf{pred pred1}(l$: seq\langle \mathbb{Z}\rangle)\ \{$\smallskip \\
			\hspace*{6mm}$body$\\
			$\}$	
			
		\item
			
			\textbf{proc intercambiarParesConImpares}(inout $l:$\textbf{String})
			$\{$\smallskip \\
			\hspace*{6mm} \textbf{Pre }$\{ \}$\smallskip \\
			\hspace*{6mm} \textbf{Post }$\{$\\
			\hspace*{6mm} $prd1(a)$\\
			\hspace*{6mm} $\wedge$\\
			\hspace*{6mm} $prd2(a)$\\
			\hspace*{6mm} $\}$\\
			$\{$\smallskip \\
			
			\textbf{pred pred1}(l$: seq\langle \mathbb{Z}\rangle)\ \{$\smallskip \\
			\hspace*{6mm}$body$\\
			$\}$	
		
		\item
		
			\textbf{proc limpiarDuplicados}(inout $l: seq\langle \mathbb{\textbf{Char}}\rangle$
			,out $dup:seq\langle \mathbb{\textbf{Char}}\rangle$)
			$\{$\smallskip \\
			\hspace*{6mm} \textbf{Pre }$\{ \}$\smallskip \\
			\hspace*{6mm} \textbf{Post }$\{$\\
			\hspace*{6mm} $prd1(a)$\\
			\hspace*{6mm} $\wedge$\\
			\hspace*{6mm} $prd2(a)$\\
			\hspace*{6mm} $\}$\\
			$\{$\smallskip \\
			
			\textbf{pred pred1}(l$: seq\langle \mathbb{Z}\rangle)\ \{$\smallskip \\
			\hspace*{6mm}$body$\\
			$\}$
	\end{enumerate}

\begin{center}
\section*{FIN.}
\end{center}

\end{document}